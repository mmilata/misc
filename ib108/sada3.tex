\documentclass[12pt]{article}
\usepackage[czech]{babel}
\usepackage[utf8]{inputenc}
\usepackage{amsmath,amssymb}
\usepackage{enumerate}
%\usepackage{hyperref}
\usepackage{algorithmic}
\usepackage{algorithm}
%\usepackage{tikz}


\pagestyle{plain}

\topmargin -5mm
\headsep 0mm
\headheight 3mm
\textheight 235mm
\textwidth 165mm
\oddsidemargin 0mm
\evensidemargin 0mm
\footskip 10mm

\newcommand{\eps}{\varepsilon}
\newcommand{\coL}{co\mbox{$-$}L}
\newcommand{\move}{\rightarrow}
\newcommand{\la}{\leftarrow}
\newcommand{\var}[1]{\textit{#1}}
\newcommand{\uv}[1]{\quotedblbase #1\textquotedblleft}
\newcommand{\ceil}[2]{\left\lceil\frac{#1}{#2}\right\rceil}
\newcommand{\floor}[2]{\left\lfloor\frac{#1}{#2}\right\rfloor}

\renewcommand{\O}{\mathcal{O}}

\newcommand{\zadani}[2]{
{\large
\noindent {\bf IB108 \hfill{} Sada #1, Příklad #2 \\[-4mm]}
\noindent\hrule
\vspace{2mm}
\noindent Vypracovali:\hfill{}Tomáš Krajča (255676), Martin Milata (256615)
\vspace{3mm}
\hrule
\bigskip\bigskip}
}


\begin{document}

\zadani{3}{1}
\noindent


\clearpage
\zadani{3}{2}
\noindent
Označme graf daných vlastností~$G$. Dále označme~$D[v]$ minimální vzdálenost
vrcholu~$v$~od~$s$, $N[v]$ počet minimálních cest do~vrcholu~$v$~z~$s$. 

\noindent
Princip algoritmu založíme na~následujícím pozorování. Do~vrcholu~$s$
z~vrcholu~$s$ vede~$0$ nejkratších 
cest délky~$0$ (záleží na~definici). 
%Všechny vrcholy 
%dosažitelné z~vrcholu~$s$ po~jedné hraně mají délku nejkratší cesty $1$ a taková
%cesta existuje právě~$1$. Mezi~dvěmi různými vrcholy v~grafu~$G$ nemůže existovat
%kratší cesta než délky~$1$, zároveň může být taková cesta nejvýše~$1$ (zřejmé).

\noindent
Nechť $u \in U = V \setminus \{s\}$ všechny ostatní vrcholy, 
 $t \in T \subseteq V$ vrcholy, které jsou z~$u$ dosažitelné po~jedné hraně, pak

$$ D[u] = \textsc{min}\{D[t]\}+1,~N[u] = \sum_{i \in T \wedge D[i] = D[u]-1} N[i]$$
Jinými slovy vezmeme všechny vrcholy, které mají minimální vzdálenost z~$s$ a
zároveň jsou dosažitelné z~$u$ po~jedné hraně. Když poté sečteme počty
minimálních cest do~těchto vrcholů, dostaneme počet minimálních cest do~vrcholu~$u$. Pokud do~vrcholu~$t$ existuje~$N[t]$ minimálních cest, pak jistě všechny
tyto cesty + hrana spojující~$t$ s~$u$ jsou stejně dlouhé. Pokud by taková cesta
nebyla minimální (jedna z minimálních), pak by ani~$D[t]$ nemohlo být minimální
(v~grafu~$G$ neexistuje mezi dvěmi různými vrcholy cesta kratší než~$1$), což je
spor s~minimalitou $D[t]$.

\noindent
Nyní stačí vyřešit problém, jak procházet vrcholy~$t$ pro~všechny vrcholy~$u$. A dále z~nich vybrat ty s~minimální~$D[t]$ a sečíst jejich~$N[t]$. K~tomu
využijeme algoritmus \textsc{BFS}, protože prochází dříve vrcholy s~menší
minimální cestou z~$s$. Hodnoty~$D[t]$ se budou grafem propagovat se
zvyšující se hodnotou. Dále pokud $D[t] < \infty$, pak už se hodnota $D[t]$ nezmění.

\begin{algorithm}
\textsc{CountSSSP}$(s, V, E)$
\begin{algorithmic}[1]
\STATE $D[v] \la \infty \mbox{ pro všechny } v \in V$ // délka nekratší cesty z~$s$ do~$v$
\STATE $N[v] \la 0 \mbox{ pro všechny } v \in V$ // počet nejkratších cest z~$s$ do~$v$
\STATE $Queue.init()$ // inicializuj frontu
\STATE $D[s] \la 0; N[s] \la 0$
\STATE $Queue.enqueue((s,\_))$; // všechny hrany z~vrcholu~$s$
\WHILE {$Queue.notempty()$}
\STATE $(u,v) \la Queue.dequeue()$
\STATE označ $(u,v)$ za~navštívenou hranu
\STATE $Queue.enqueue((v,\_)) \mbox{, kde } (v,\_)$ je nenavštívená hrana
\IF {$D[u] < D[v]$}
\STATE $D[v] \la D[u]+1$; $N[v] \la N[u] + N[v]$
\ENDIF
\ENDWHILE
\RETURN $N$
\end{algorithmic}
\end{algorithm}
Řádky $1$,$2$ mají cenu~$\O(|V|)$. V~každém průchodu cyklem
\textit{while} ($6$-$13$) se označí jedna hrana grafu~$G$ za~navštívenou. Tato
hrana už se nikdy nepřidá do~fronty. Fronta se tedy musí nejvýše po~$|H|$
průchodech cyklem vyprázdnit. Algoritmus skončí. Výsledná složitost je~$\O(|V|+|H|)$.

\hfill$\square$

\clearpage
\zadani{3}{3}

\clearpage
\zadani{3}{4}

\noindent
Změnu ohodnocení hrany můžeme rozdělit na čtyři případy podle toho, zda je daná hrana součástí
minimální kostry v původním grafu a zda-li se její cena zvýšila nebo snížila.  Snadno nahlédneme, že
zvýší-li se cena hrany, která není součástí minimální kostry, minimální kostra se nezmění. Stejně
tak se nezmění kostra v případě, že se sníží cena hrany, která do kostry patří.

Pokud se zvýšila cena hrany ležící v původní kostře, odstraníme ji. Tím nám vzniknou dva stromy,
které jsou na svých vrcholech minimální
(pokud by nebyly minimální, byl by to spor s minimalitou původní kostry). Novou kostru
vytvoříme přidáním nejlevnější hrany, která oba vzniklé stromy opět spojí do jednoho (argument
korektnosti je stejný jako u Kruskalova algoritmu).

Podobně, pokud se snížila cena hrany neležící v původní kostře, do kostry ji přidáme. Tím vznikne graf
s právě jedním cyklem (pokud by vzniklo více cyklů, znamenalo by to, že před přidáním hrany už tam nějaký
cyklus byl, a že se tedy nejednalo o strom). Z tohoto cyklu poté odebereme hranu s nejvyšší cenou. Tento
postup je korektní, protože pokud bychom odstranili celý cyklus, získáme les stromů, které jsou
minimální na svých vrcholech.
Ty poté spojujeme postupným přidáním hran na cyklu (kdybychom museli přidat hranu která leží mimo
tento cyklus, dostali bychom spor s minimalitou původní kostry), které přidáme všechny kromě té nejdražší
-- tu tedy stačí odstranit.

Nechť $V$ je množina vrcholů, $E$ množina hran, $E'$ původní minimální kostra, $e=(u,v)$ hrana, která
se změnila, $w$ původní ohodnocení (cena) hrany a $w'$ nové ohodnocení hrany $e$.

\begin{algorithm}
\textsc{UpdateMST}$(V, E, E', e, w, w')$
\begin{algorithmic}[1]
\IF{$e \in E' \wedge w' < w$}
	\RETURN $(V,E')$
\ELSIF{$e \in E' \wedge w' > w$}
	\STATE $E' \la E' \smallsetminus \{e\}$
	\STATE Z vrcholu $u$ spusť na grafu $(V,E')$ \textsc{Dfs}, které všechny dosažené vrcholy označí $1$.
	\STATE Z vrcholu $v$ spusť na grafu $(V,E')$ \textsc{Dfs}, které všechny dosažené vrcholy označí $2$.
	\STATE Projdi všechny hrany takové, že jejich vrcholy jsou označeny různými čísly a vyber z~nich hranu $f$ s nejmenší cenou.
	\RETURN $(V,E' \cup \{f\})$
\ELSIF{$e \not\in E' \wedge w' < w$}
	\STATE $E' \la E' \cup \{e\}$
	\STATE Pomocí \textsc{Dfs} z vrcholu $u$ najdi v $(V,E')$ cyklus, hrany na něm ulož do množiny $C$.
	\STATE Nechť $f$ je hrana z $C$ s nejvyšší cenou.
	\RETURN $(V,E' \smallsetminus \{f\})$
\ELSIF{$e \not\in E' \wedge w' > w$}
	\RETURN $(V,E')$
\ENDIF
\end{algorithmic}
\end{algorithm}

\clearpage
\zadani{3}{4}
\noindent
Algoritmus realizuje výše uvedený postup. Prohledávání do hloubky trvá $\O(|E|)$, stejně tak procházení množin hran
(pokud jsou v nějaké rozumné datové struktuře, například pokud je graf reprezentován maticí sousednosti). Složitost
algoritmu je tedy $\O(|E|)$. Konvergence algoritmu vyplývá z konvergence podprogramů a absence
cyklů.

\end{document}
