\documentclass[12pt]{article}
\usepackage[czech]{babel}
\usepackage[utf8]{inputenc}
\usepackage{amsmath,amssymb}
\usepackage{enumerate}
%\usepackage{hyperref}
\usepackage{algorithmic}
\usepackage{algorithm}
%\usepackage{tikz}


\pagestyle{plain}

\topmargin -5mm
\headsep 0mm
\headheight 3mm
\textheight 235mm
\textwidth 165mm
\oddsidemargin 0mm
\evensidemargin 0mm
\footskip 10mm

\newcommand{\eps}{\varepsilon}
\newcommand{\coL}{co\mbox{$-$}L}
\newcommand{\move}{\rightarrow}
\newcommand{\la}{\leftarrow}
\newcommand{\var}[1]{\textit{#1}}
\newcommand{\uv}[1]{\quotedblbase #1\textquotedblleft}
\newcommand{\ceil}[2]{\left\lceil\frac{#1}{#2}\right\rceil}
\newcommand{\floor}[2]{\left\lfloor\frac{#1}{#2}\right\rfloor}

\renewcommand{\O}{\mathcal{O}}

\newcommand{\zadani}[2]{
{\large
\noindent {\bf IB108 \hfill{} Sada #1, Příklad #2 \\[-4mm]}
\noindent\hrule
\vspace{2mm}
\noindent Vypracovali:\hfill{}Tomáš Krajča (255676), Martin Milata (256615)
\vspace{3mm}
\hrule
\bigskip\bigskip}
}


\begin{document}

\zadani{3}{1}

\clearpage
\zadani{3}{2}

\clearpage
\zadani{3}{3}

\clearpage
\zadani{3}{4}

\noindent
Změnu ohodnocení hrany můžeme rozdělit na čtyři případy podle toho, zda je daná hrana
součástí minimální kostry v původním grafu a podle toho, zda-li se její cena zvýšila nebo snížila.
Snadno nahlédneme, že zvýší-li se cena hrany, která není součástí minimální kostry, minimální kostra
se nezmění. Stejně tak se nezmění kostra v případě, že se sníží cena hrany, která do kostry patří.

Pokud se zvýšila cena hrany ležící v původní kostře, odstraníme ji. Tím nám vzniknou dva minimální
stromy (pokud by nebyly minimální, byl by to spor s minimalitou původní kostry). Novou kostru vytvoříme
přidáním nejlevnější hrany, která oba vzniklé stromy opět spojí do jednoho (argument korektnosti je
stejný jako u Kruskalova algoritmu).

Podobně, pokud se snížila cena hrany neležící v původní kostře, do kostry ji přidáme. Tím vznikne graf
s právě jedním cyklem (pokud by vzniklo více cyklů, znamenalo by to, že před přidáním hrany už tam nějáký
cyklus byl, a že se tedy nejednalo o strom). Z tohoto cyklu poté odebereme hranu s nejvyšší cenou. Tento
postup je korektní, protože pokud bychom odstranili celý cyklus, získáme les minimálních stromů.
Ty poté spojujeme postupným přidáním hran na cyklu (kdybychom museli přidat hranu která leží mimo
tento cyklus, dostali bychom spor s minimalitou původní kostry), které přidáme všechny kromě té nejdražší
-- tu tedy stačí odstranit.

Nechť $V$ je množina vrcholů, $E$ množina hran, $E'$ původní minimální kostra, $e=(u,v)$ hrana, která
se změnila, $w$ původní ohodnocení (cena) hrany a $w'$ nové ohodnocení hrany $e$.

\begin{algorithm}
\textsc{UpdateMST}$(V,E,E',e=(u,v),w,w')$
\begin{algorithmic}[1]
\IF{$e \in E' \wedge w' < w$}
	\RETURN $(V,E')$
\ELSIF{$e \in E' \wedge w' > w$}
	\STATE $E' \la E' \smallsetminus \{e\}$
	\STATE Z vrcholu $u$ spusť \textsc{Dfs}, které všechny dosažené vrcholy označí $1$.
	\STATE Z vrcholu $v$ spusť \textsc{Dfs}, které všechny dosažené vrcholy označí $2$.
	\STATE Projdi všechny hrany takové, že jejich vrcholy jsou označeny různými čísly a vyber z nich hranu $f$ s nejmenší cenou.
	\RETURN $(V,E' \cup \{f\})$
\ELSIF{$e \not\in E' \wedge w' < w$}
	\STATE $E' \la E' \cup \{e\}$
	\STATE Pomocí \textsc{Dfs} z vrcholu $u$ najdi cyklus, hrany na něm ulož do množiny $C$.
	\STATE Nechť $f$ je hrana z $C$ s nejvyšší cenou.
	\RETURN $(V,E' \smallsetminus \{f\})$
\ELSIF{$e \not\in E' \wedge w' > w$}
	\RETURN $(V,E')$
\ENDIF
\end{algorithmic}
\end{algorithm}

%\clearpage
%\zadani{3}{4}
\noindent
Algoritmus realizuje výše uvedený postup. Prohledávání do hloubky trvá $\O(|E|)$, stejně tak procházení množin hran
(pokud jsou v nějáké rozumné datové struktuře, například pokud je graf reprezentován maticí sousednosti). Složitost
algoritmu je tedy $\O(|E|)$.

\end{document}
