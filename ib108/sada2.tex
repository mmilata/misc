\documentclass[12pt]{article}
\usepackage[czech]{babel}
\usepackage[utf8]{inputenc}
\usepackage{amsmath,amssymb}
\usepackage{enumerate}
%\usepackage{hyperref}
\usepackage{algorithmic}
\usepackage{algorithm}
%\usepackage{tikz}

\pagestyle{plain}

\topmargin -5mm
\headsep 0mm
\headheight 3mm
\textheight 235mm
\textwidth 165mm
\oddsidemargin 0mm
\evensidemargin 0mm
\footskip 10mm

\newcommand{\eps}{\varepsilon}
\newcommand{\coL}{co\mbox{$-$}L}
\newcommand{\move}{\rightarrow}
\newcommand{\la}{\leftarrow}
\newcommand{\var}[1]{\textit{#1}}

\renewcommand{\O}{\mathcal{O}}

\newcommand{\zadani}[2]{
{\large
\noindent {\bf IB108 \hfill{} Sada #1, Příklad #2 \\[-4mm]}
\noindent\hrule
\vspace{2mm}
\noindent Vypracovali:\hfill{}Tomáš Krajča (255676), Martin Milata (256615)
\vspace{3mm}
\hrule
\bigskip\bigskip}
}


\begin{document}

\zadani{2}{1}
% TODO:
% obě poloviny mají lichý počet studentů
% pole studentů obsahuje přesně dvě poloviny spolužáků

\noindent
%Nechť $d:student \times student \move \{yes, no\}$ je dotaz do databáze.
\begin{equation*}
\textsc{Schoolmates} = \left\{
	\begin{array}{rl}
	0 & \text{pole studentů neobsahuje alespoň $\left\lceil \frac{n}{2}
	\right\rceil$ spolužáků}\\
	1 & \text{\var{students}[1..t] jsou spolužáci $\mid t \geq \left\lceil \frac{n}{2}
	\right\rceil$}\\
	2 & \text{\var{students}[1..$\left\lceil \frac{n}{2}
	\right\rceil$], \var{students}[$\left\lceil \frac{n}{2}
	\right\rceil+1$..N] jsou
	spolužáci (avšak různí)}\\
	\end{array} \right.
\end{equation*}

\begin{algorithm}
\textsc{Schoolmates}($students[1..N]$)
\begin{algorithmic}
\IF {$N > 2$}
\STATE $Schoolmates1 \gets$ \textsc{Schoolmates}($students[1..\left\lfloor \frac{N}{2}
\right\rfloor]$)
\STATE $Schoolmates2 \gets$ \textsc{Schoolmates}($students[(\left\lfloor \frac{N}{2} \right\rfloor+1)..N]$)
\IF{Schoolmates1 i Schoolmates2 obsahují alespoň N/2 spolužáků \hfill(1)}
\STATE máme maximálně 4 různé skupiny spolužáků ($\geq N/2$), srovnáme
jednotlivé vzorky, pokud se rovnají přeskládáme je tak, aby první část pole byli
spolužáci, pokud se rovnají i další 2 skupiny, vrátíme $2$, jinak vrátíme $1$.
Pokud se nerovnají alespoň $2$ skupiny, vrátíme $0$.
\ELSIF{Schoolmates1 nebo Schoolmates2 obsahují alespoň N/2 spolužáků\hfill(2)}
\STATE Postupně porovnáme všechny studenty z části pole, kde není alespoň $N/2$
spolužáků (pokud najdeme spolužáka, umístíme ho na konec N/2 spolužáků). Pokud
najdeme alespoň $<N/2$ dalších spolužáků, vrátíme $1$, jinak vrátíme $0$.
\ELSIF{Schoolmates1 ani Schoolmates2 neobsahují alespoň N/2 spolužáků \hfill(3)}
\RETURN $0$
\ENDIF
\ELSE
\RETURN $2$
\ENDIF
\end{algorithmic}
\end{algorithm}

\noindent
\textbf{Složitost} - (1) maximálně $4$ dotazy\\
                     (2) maximálně $N/2$ dotazů\\
		     (3) $0$ dotazů\\
		     Tedy $T(0) = c, T(n) \leq T(n/2)+T(n/2) + n/2$\\
		     Dle Master Theorem $\mathcal{O}(n\log n)$\\
\textbf{Korektnost} - Indukcí k délce paramteru pole
\begin{itemize}
\item Báze - pole délky $1$, $2$ - triviální - žák je sám sobě spolužákem
\item Předpoklad - funguje pro pole délky $n$
\item Krok - existují následující $3$ možnosti:\begin{itemize}
\item (1) - dikuze jednotlivých možností
\item (2) - opět diskuze
\item (3) - v obou slučovaných polích je ($<N/2$) spolužáků, maximálně by tedy
mohlo ve sloučeném poli mohlo být $N/2-1$ spolužáků.
\end{itemize}
\end{itemize}
\textbf{Konvergence} - při rekurzivním volání funkce \textsc{Schoolmates} vždy
snížíme velikost pole, na které je volána. Zřejmě v konečném počtu kroků musí
velikost pole klesnout pod hodnotu $2$. 
\hfill$\square$


\clearpage
\zadani{2}{2}

% TODO:
% - odstranit ten algoritmus? je docela zbytecny ...
% - vyresit kolize v namespacu promennych

\noindent
Zřejmě nezáleží na čase, kdy závodník běží a kdy jede na kole, záleží pouze na jejich součtu.
Označme si proto $s_i = k_i + b_i$.  Problém má optimální substrukturu, kterou můžeme
charakterizovat funkcí pro výpočet minimální délky závodu:
$$t(\emptyset) = 0$$
$$t(I) = 
\min_{i \in I} \left(
		p_i + \max \left\{
				s_i, t(I\smallsetminus\left\{i\right\})
		\right\}
\right)$$

Ukážeme, že při využití hladového kritéria podle kterého vybíráme jako prvního závodníka s největším
$s_i$ získáme optimální výsledek. Tedy že $t(I) = p_m + max(s_m, t(I \smallsetminus \{m\}))$, kde $m
\in I$ je takové, že $s_m$ je maximální.

Algoritmus, který do pole $A$ zapíše optimální pořadí závodníků a spočítá čas, kdy doběhne poslední,
může vypadat následovně.

\begin{algorithm}
\begin{algorithmic}
\STATE $A \la$ indexy závodníků seřazené \textsc{MergeSort}em sestupně podle $s_i$
\STATE $pmax \la 0$
\STATE $smax \la 0$
\FOR{$i \la 1 $ to $N$}
\STATE $pmax \la pmax + p_{A[i]}$
\STATE $smax \la \max(pmax+s_{A[i]},smax)$
\ENDFOR
\RETURN $smax$
\end{algorithmic}
\end{algorithm}

\noindent
\textsc{MergeSort} má časovou složitost $\O(N\log N)$, for cyklus se provede pouze $N$-krát.
Celkově je tedy algoritmus v $\O(N\log N)$.

\bigskip

\noindent
Nyní chceme dokázat, že posloupnost $S = \langle i_1,i_2,...,i_N \rangle$ indexů závodníků získaná tímto
algoritmem, tedy seřazená sestupně podle $s_i$ je optimální.

Definujme si funkci, která pro danou permutaci indexů závodníků určí trvání závodu (říkejme mu cena),
%$$ v(a_1,a_2,\cdots,a_m) = \sum_{i=1}^N p_{a_i} + \max_{i=1}^N \left(s_{a_i} - \Sigma_{j=i}^N p_{a_j}\right)$$
$$ v(\langle a_1,a_2,\cdots,a_m \rangle) = \sum_{i=1}^m p_{a_i} + \max_{i=1}^m u(i)$$
$$ u(i) = s_{a_i} - \sum_{j=i+1}^m p_i$$
tedy celkové trvání, kdy budou závodníci plavat plus nejdelší čas, který nějáký závodník poběží od
chvíle, kdy všichni doplavali. Funkce $u(i)$ říká kolik času uběhlo mezi tím co doplaval poslední
závodník a mezi tím, co závodník $a_i$ doběhl do cíle. Její hodnota může být záporná, nicméně
alespoň pro index $m$ bude vždy kladná.

\clearpage
\zadani{2}{2}

\noindent
Předpokládejme existenci posloupnosti $T = \langle j_1,j_2,...,j_N \rangle$, pro kterou platí $S
\neq T$ a jejíž cena je optimální. Protože $S \neq T$, existují indexy $j_k, j_{k+1}$ takové, že
$s_{j_k} < s_{j_{k+1}}$. Ukážeme, že prohodíme-li tyto dva prvky posloupnosti (kterou si označíme
$T'$), její cena se nezvýší.

Označme si $u'(i)$ hodnotu $u(i)$ na takto vzniklé posloupnosti. Zřejmě platí, že $u'(i) = u(i)$ pro
$i \notin \{k, k+1\}$ -- pro $i < k$ se změní pouze pořadí indexů větší než $i$, ne součet jejich
$p$; pro $i > k+1$ pak $p_{j_k}, p_{j_{k+1}}$ nejsou ani v sumě obsaženy. Dále platí:

$$
\begin{array}{rclclcl}
u(k)    & = & s_{j_k}     & - & p_{j_{k+1}} & - & \Sigma_{j=k+2}^N p_j \\
u(k+1)  & = & s_{j_{k+1}} &     &               & - & \Sigma_{j=k+2}^N p_j \\
u'(k)   & = & s_{j_k}     &     &               & - & \Sigma_{j=k+2}^N p_j \\
u'(k+1) & = & s_{j_{k+1}} & - & p_k         & - & \Sigma_{j=k+2}^N p_j \\
\end{array}
$$

\noindent
Protože $s_{j_k} < s_{j_{k+1}}$, vyplývá z těchto rovnic, že $u'(k) < u(k+1)$ a také, že $u'(k+1) <
u(k+1)$. Takže $\max u'(i)$ nebude větší než $\max u(i)$, cena se nezvýší.

\medskip
\noindent
Získáváme tedy jeden ze tří případů:
\begin{itemize}
\item Cena takto změněné posloupnosti je menší. To je spor s optimalitou $T$.
\item Cena zůstala stejná a $S = T'$. To znamená, že i $S$ je optimální.
\item Cena zůstala stejná a $S \neq T'$. Pak opět můžeme prohodit dva prvky, což děláme tak dlouho,
      dokud nedostaneme jeden z předchozích dvou případů.
\end{itemize}

%Zavedeme si funkci $v$, která nám pro
%danou posloupnost závodníků určí celkové trvání závodu a říkejme tomuto trvání cena.
%
%$$ v(\langle \rangle) = 0 $$
%$$ v(\langle a_1,a_2,...,a_m \rangle) = p_{a_1} + \max \{s_{a_1}, v(\langle a_2,...,a_m \rangle)\} $$
%
%\noindent
%Předpokládejme existenci posloupnosti $T = \langle j_1,j_2,...,j_N \rangle$, pro kterou platí $S
%\neq T$ a která je optimální. %Podle definice $v$ existuje $k$ takové, že
%%$$ v(T) = p_{j_1} + p_{j_2} + ... + p_{j_k} + s_{j_k}. $$


\clearpage
\zadani{2}{3}



\clearpage
\zadani{2}{4}



\clearpage
\zadani{2}{5}

\noindent
Algoritmus ověří, zda je aktuálě zpracovávaný vrchol extremální a pokud není, znamená to, že kořen
alespoň jednoho podstromu má menší hodnotu. Na tento podstrom se algoritmus rekurzivně zavolá (pokud
mají menší hodnotu vrcholy obou podstromů, zavolá se na levý). Takto vždy dojde k extremálnímu
vrcholu -- viz níže.

%\begin{algorithm}
%\begin{algorithmic}
%\STATE $node \la root$
%\WHILE{not $\textsc{Leaf}(node)$}
%\IF{$\textsc{Value}(node.left) < \textsc{Value}(node)$}
%\STATE $node \la node.left$
%\ELSIF{$\textsc{Value}(node.right) < \textsc{Value}(node)$}
%\STATE $node \la node.right$
%\ELSE
%\RETURN $node$
%\ENDIF
%\ENDWHILE
%\RETURN $node$
%\end{algorithmic}
%\end{algorithm}

\begin{algorithm}
\textsc{FindExtremal}($node$)
\begin{algorithmic}
\IF{$\textsc{IsLeaf}(node)$}
\RETURN $node$
\ELSIF{$\textsc{Value}(node.left) < \textsc{Value}(node)$}
\RETURN \textsc{FindExtremal}($node.left$)
\ELSIF{$\textsc{Value}(node.right) < \textsc{Value}(node)$}
\RETURN \textsc{FindExtremal}($node.right$)
\ELSE
\RETURN $node$
\ENDIF
\end{algorithmic}
\end{algorithm}

Invariantem funkce \textsc{FindExtremal} je, že rodič aktuálně zpracovávaného vrcholu je větší než
on sám (nebo je zpracovávaný rodič kořen). V kořeni triviálně platí a rekurzivní volání na podstrom
s menší hodnotou kořene jej zachovávají. Pokud platí invariant a oba podstromy mají kořen s větší
hodnotou, je vrchol extremální. Podobně, pokud platí invariant a zpracovávaný vrchol je listem, je
tento list menší než všechny vrcholy a je tedy rovněž extremální.

Z toho, že se algoritmus rekurzivně volá na jeden z podstromů, kterýžto má o jedna menší výšku,
vyplývají následující fakta:

\begin{itemize}
\item Algoritmus je korektní, protože vždy dojde buď do listu nebo předtím narazí na vnitřní
extremální vrchol.
\item Algoritmus je konvergentní.
\item Protože se v každé instanci volá \textsc{Value} nejvýše čtyřikrát\footnote{počet těchto volání
by bylo možné snížit na dvě} a těchto instancí je nejvýše tolik, kolik je výška stromu, tedy $\log
n$, je počet volání $\O(\log n)$.
\end{itemize}

\hfill$\square$

\end{document}
