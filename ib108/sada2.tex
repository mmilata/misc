\documentclass[12pt]{article}
\usepackage[czech]{babel}
\usepackage[utf8]{inputenc}
\usepackage{amsmath,amssymb}
\usepackage{enumerate}
%\usepackage{hyperref}
\usepackage{algorithmic}
\usepackage{algorithm}
%\usepackage{tikz}


\pagestyle{plain}

\topmargin -5mm
\headsep 0mm
\headheight 3mm
\textheight 235mm
\textwidth 165mm
\oddsidemargin 0mm
\evensidemargin 0mm
\footskip 10mm

\newcommand{\eps}{\varepsilon}
\newcommand{\coL}{co\mbox{$-$}L}
\newcommand{\move}{\rightarrow}
\newcommand{\la}{\leftarrow}
\newcommand{\var}[1]{\textit{#1}}
\newcommand{\uv}[1]{\quotedblbase #1\textquotedblleft}
\newcommand{\ceil}[2]{\left\lceil\frac{#1}{#2}\right\rceil}
\newcommand{\floor}[2]{\left\lfloor\frac{#1}{#2}\right\rfloor}

\renewcommand{\O}{\mathcal{O}}

\newcommand{\zadani}[2]{
{\large
\noindent {\bf IB108 \hfill{} Sada #1, Příklad #2 \\[-4mm]}
\noindent\hrule
\vspace{2mm}
\noindent Vypracovali:\hfill{}Tomáš Krajča (255676), Martin Milata (256615)
\vspace{3mm}
\hrule
\bigskip\bigskip}
}


\begin{document}

\zadani{2}{1}
% snad ne příliš zmatené

\noindent
%Nechť $d:student \times student \move \{yes, no\}$ je dotaz do databáze.
Nechť pole \var{students[1..N]} reprezentuje soubor studentů.
Nechť výsledek funkce \textsc{Schoolmates} je kódovaný takto:
\begin{equation*}
\textsc{Schoolmates}(L,R) = \left\{
	\begin{array}{rl}
	true & \text{alespoň $\left\lceil\frac{R-L+1}{2}\right\rceil$ studentů
	jsou spolužáci}\\
	false & \text{jinak}
	\end{array} \right.
\end{equation*}
Algoritmus využívá toho, že pokud známe výsledek pro~$2$ podobně dlouhé
podposloupnosti, relativně snadno určíme výsledek i pro~jejich sloučení. Invariantem
algoritmu je struktura souboru studentů. Pokud soubor obsahuje, alespoň $\ceil{N}{2}$
spolužáků, pak jsou všichni tito spolužáci v~první části pole za~sebou.

\begin{algorithm}
\textsc{Schoolmates}($L,R$)
\begin{algorithmic}[1]
\STATE $N \la R-L+1$
\IF {$N > 2$}
\STATE \textsc{Schoolmates}($L,\left\lfloor\frac{L+R}{2} \right\rfloor$)
\STATE \textsc{Schoolmates}($\left\lfloor\frac{L+R}{2} \right\rfloor+1,R$)
\IF {\textsc{TF}($L,R,N$)}
\STATE přeskládej studenty tak, aby první část pole \var{students[L..R]} byli
všichni spolužáci, kterých je alespoň $\left\lceil\frac{N}{2} \right\rceil$
\RETURN true
\ELSE
\RETURN false
\ENDIF

%\STATE $Schoolm1 \gets$ \textsc{Schoolmates}($L,\left\lfloor \frac{R}{2}
%\right\rfloor$); $Schoolm2 \gets$ \textsc{Schoolmates}($\left\lfloor \frac{R}{2} \right\rfloor+1,R$)
%\IF{$Schoolm1 = 2 \vee Schoolm2 = 2$}
%\STATE Určíme tu část pole, která vrátila výsledek $2$. BÚNO je to
%$Schoolm1$.
%\STATE Postupně porovnáváme všechny prvky z části pole $Schoolm2$ s vybraným
%žákem z první a poté druhé části pole $Schoolm1$.
%\IF {najdeme dalších $\frac{N}{4}$ spolužáků první části $\wedge$ dalších
%$\frac{N}{4}$ spolužáků druhé části $\mid N$ sudé}
%\STATE přeskládáme studenty tak, aby první polovina spojeného pole
%byli spolužáci
%\RETURN $2$
%\ELSIF {najdeme dalších alespoň $\left\lceil\frac{N}{4}\right\rceil$
%spolužáků}
%\STATE přeskládáme studenty tak, aby první část spojeného pole tvořili tito spolužáci
%\RETURN $1$
%\ELSE
%\RETURN $0$
%\ENDIF
%\ELSIF{$Schoolm1 = 1 \vee Schoolm2 = 1$}
%\STATE Určíme tu část pole, která vrátila výsledek $1$ a počet spolužáků v její
%první části. BÚNO je to
%$Schoolm1$ a $x$.
%\STATE Postupně porovnáme všechny studenty z části pole $Schoolm2$ s
%vybraným žákem z první části pole $Schoolm2$
%\IF{najdeme dalších alespoň $y$ spolužáků $\mid y+x\geq
%\left\lceil\frac{N}{2}\right\rceil$}
%\STATE přeskládáme studenty tak, aby první část spojeného pole tvořili tito spolužáci
%\RETURN $1$
%\ELSE
%\RETURN $0$
%\ENDIF
%\ELSIF{$Schoolm1 = 0 \wedge Schoolm2 = 0$}
%\RETURN $0$
%\ENDIF

\ELSE
\RETURN true
\ENDIF
\end{algorithmic}
\end{algorithm}
Operace \uv{přeskládání studentů} -- jistě umíme studenty přeskládat za~použítí
lineárního počtu dotazů do~databáze vzhledem k~délce sloučeného souboru~$N$. Známe
reprezentanta skupiny spolužáků (je v~první polovině podsouboru, na~které byla
rekurentně volána funkce \textsc{Schoolmates} s~výsledkem $true$), které chceme přeskládat do~první části
výsledného souboru. Můžeme tedy s~tímto reprezentantem srovnat všechny žáky
z~jednoho podsouboru a pak všechny žáky ze~druhého podsouboru. Příslušné spolužáky pak
přeskládáme do~první části sloučeného souboru. Algoritmus je zřejmě korektní (dokázalo by se
indukcí k~délce souboru~$N$).

\clearpage
\zadani{2}{1}

\begin{equation*}
\textsc{TF}(L,R,N) = \left\{
	\begin{array}{cl}
	true & \text{$(t-L+1)+f \geq \left\lceil\frac{N}{2}\right\rceil \mid
	students[L..t]$ spolužáci,} \\ 
	& \text{f počet jejich spolužáků ve
	$students[\left\lfloor\frac{L+R}{2}\right\rfloor+1$..R]}\\

	true & \text{$(t-(\floor{L+R}{2}+1)+1)+f \geq
	\left\lceil\frac{N}{2}\right\rceil \mid$}\\
	&\text{$students[\floor{L+R}{2}+1..t$] spolužáci,} \\ 
	& \text{f počet jejich spolužáků ve
	$students[L..\left\lfloor\frac{L+R}{2}\right\rfloor$]}\\

	true &
	\text{$students[\left\lfloor\frac{L+R}{4}\right\rfloor+1..\floor{R+L}{2}]
	$ spolužáci $\wedge \floor{N}{4}+f \geq \left\lceil\frac{N}{2}\right\rceil \mid$}\\
	& \text{f počet jejich spolužáků ve
	$students[\left\lfloor\frac{L+R}{2}\right\rfloor+1$..R]}\\

	true & \text{$students[\left\lfloor\frac{3(L+R)}{4}\right\rfloor+1..R]
	$ spolužáci $\wedge \floor{N}{4}+f \geq \left\lceil\frac{N}{2}\right\rceil \mid$}\\
	& \text{f počet jejich spolužáků ve
	$students[L..\floor{L+R}{2}$]}\\

	false & \text{jinak}
	\end{array} \right.
\end{equation*}

\noindent
Důkaz, že funkce správně vrací hodnotu $true$, je triviální (plyne z~definice
funkce pro~výsledky $true$). Sporem dokážeme, že v~ostaních případech správně
vrací $false$. 

\noindent
Nechť tedy existuje podsoubor $A$ a podsoubor $B$, jejichž sloučením je
alespoň $\ceil{N}{2}$ spolužáků a přitom by funkce \textsc{TF} vrátila $false$. Má-li
výsledný soubor obsahovat alespoň $\ceil{N}{2}$ spolužáků, musí $A$ nebo $B$ obsahovat
alespoň $\ceil{N}{4}$ spolužáků ($\ceil{N}{4}+(\ceil{N}{4}-1)<\ceil{N}{2}$).
BÚNO je to $A$ a těchto spolužáku je $a$ ($A$ může dokonce obsahovat $2$ takové
skupiny). Pak tedy $B$ musí obsahovat $b \geq
\ceil{N}{2}-a$ jejich spolužáků ($b \geq 0$). V~tom případě, ale funkce
\textsc{TF}
vrátí $true$ ($1.$ nebo $3.$ případ). Symetricky pokud předpokládáme, že
podsoubor $B$
obsahuje alespoň $\ceil{N}{2}$ spolužáků ($2.$ nebo $4.$ případ). Což je
spor s~tím, že existuje $A$ a $B$.

\noindent
Pro~zjištění $t$ (\var{students[i..t]} spolužáci) nám stačí lineární počet
dotazů do~databáze vzhledem k~velikosti souboru~$N$. Jednoduše porovnáme
\var{students[i]} se \var{students[i+1]}, \var{students[i+1]} se
\var{students[i+2]} atd. až najdeme
první neshodu (popř. konec intervalu). Nechť první neshoda je při porovnání
\var{students[n]} se \var{students[n+1]},
pak $t=n$. Algoritmus je zřejmě korektní (dokázalo by se indukcí vzhledem
k~délce souboru $N$).

\noindent
Pro zjištění počtu spolužáků vybraného žáka v~souboru o délce $N$ nám zřejmě
opět stačí lineární počet dotazů do~databáze. Porovnáme všechny žáky souboru s~vybraným
žákem. Algoritmus je zřejmě korektní (dokázalo by se indukcí k~délce pole $N$).
%\begin{equation*}
%\textsc{TF}(true,false) = \left\{
%	\begin{array}{cl}
%	true & \text{$(t-L)+f \geq \left\lceil\frac{N}{2}\right\rceil \mid$
%	students[L..t] spolužáci,} \\ 
%	& \text{f počet jejich spolužáků ve
%	students[$\left\lfloor\frac{L+R}{2}\right\rfloor+1$,R]}\\
%	% zde je zachyceno i pokud jsou spoluzaci cele podpole
%	% pokud je druha pulka taky spoluzaci, tak protoze ve druhem poli neni
%	%+ani polovina studentu spoluzaci, tedy dohromady taky nemuzou byt
%	%+spoluzaci
%	false & \text{jinak}
%	\end{array} \right.
%\end{equation*}

%\begin{equation*}
%\textsc{TF}(false,true) = \left\{
%	\begin{array}{cl}
%	true & \text{$(R-t)+f \geq \left\lceil\frac{N}{2}\right\rceil \mid$
%	students[t..R] spolužáci,} \\ 
%	& \text{f počet jejich spolužáků ve
%	students[L..$\left\lfloor\frac{L+R}{2}\right\rfloor$]}\\
%	false & \text{jinak}
%	\end{array} \right.
%\end{equation*}

%\begin{equation*}
%\textsc{TF}(false,false) = false
%\end{equation*}

\clearpage
\zadani{2}{1}
\noindent
Analýza algoritmu \textsc{Schoolmates}:\\
\textbf{Složitost}\footnote{Počet dotazů do~databáze by bylo možné
optimalizovat.}\\
Složitost funkce \textsc{TF} $\in \mathcal{O}(n)$, dále jen operace s~konstantní
složitostí (kromě rekurentního volání \textsc{Schoolmates}).
Tedy rekurentní rovnice pro funkci \textsc{Schoolmates}, kde parametr funkce $T$ udává počet
studentů v souboru je $T(1) \leq c, T(2) \leq c, T(N) \leq 2\cdot T(\ceil{N}{2}) +
c\cdot N) \mid c \text{ vhodná konstanta}$.\\
Dle \textit{Master Theorem} je tedy počet dotazů $\in \mathcal{O}(N\log N)$.\\
\textbf{Korektnost}\\
Důkaz indukcí k~počtu studentů (délce souboru $N = R - L + 1$).
\renewcommand{\labelenumi}{\textbf{\alph{enumi})}}
\begin{itemize}
\item \textbf{Báze} -- počet studentů $N=1$, $N=2$ -- triviální -- žák je sám
sobě spolužákem, v~první části souboru studentů je alespoň $\ceil{N}{2}$
spolužáků.
\item \textbf{Krok} -- počet studentů $N$. Z~indukčního předpokladu víme, že funkce
\textsc{Schoolmates} funguje správně pro~soubory délky $< N$. 
%Pokud tedy jedno
%ze dvou jejích rekurentních volání vrátilo true, bude platit invariant o
%struktuře příslušného podpole. 
Funkce \textsc{TF} tedy správně vrátí $true$ nebo
$false$ (platí invariant o struktuře podsouborů dle IP). Čímž jsme dokázali i platnost invariantu o~struktuře souboru.
\end{itemize}
\textbf{Konvergence}\\ 
Při~rekurzivním volání funkce \textsc{Schoolmates} na~řádku $3$ a $4$ vždy
snížíme velikost souboru, na~které je volána. Zřejmě v~konečném počtu kroků musí
parametr~$N$ klesnout pod~hodnotu~$3$. 
\hfill$\square$



\clearpage
\zadani{2}{2}

% TODO:
% - odstranit ten algoritmus? je docela zbytecny ...
% - pripsat nejaky komentar k hornim rovnicim (a mozna i k dolnim)

\noindent
Zřejmě nezáleží na čase, kdy závodník běží a kdy jede na kole, záleží pouze na jejich součtu.
Označme si proto $s_i = k_i + b_i$.  Problém má optimální substrukturu, kterou můžeme
charakterizovat funkcí pro výpočet minimální délky závodu ($I$ je množina indexů označujících
účastníky závodu):
$$t(\emptyset) = 0$$
$$t(I) = 
\min_{i \in I} \left(
		p_i + \max \left\{
				s_i, t(I\smallsetminus\left\{i\right\})
		\right\}
\right)$$

První rovnice říká, že závod prázdné množiny závodníků trvá nula minut. Druhá pak, že je-li množina
neprázdná, pak je délka závodu rovna délce času, který bude nějaký závodník plavat plus délka času,
kdy bude závodit na suché trati nebo délka času, kterou by trval závod ostatních -- podle
toho, co trvá déle. ,,Nějáký závodník`` v tomto případě znamená, že z možných součtů vybíráme
optimum.

Ukážeme, že při využití hladového kritéria podle kterého vybíráme nejdříve závodníky s největším
$s_i$ získáme optimální výsledek. Tedy že $t(I) = p_m + max(s_m, t(I \smallsetminus \{m\}))$, kde $m
\in I$ je takové, že $s_m$ je maximální.

Algoritmus pro nalezení optimálního pořadí závodníků je tedy pouze sestupně seřadí podle $s_i$ -- na
to můžeme použít některý ze známých řadících algoritmů, například \textsc{MergeSort} se složitostí
$\O(N\log N)$.

%Algoritmus, který do pole $A$ zapíše optimální pořadí závodníků a spočítá čas, kdy doběhne poslední,
%může vypadat následovně.
%
%\begin{algorithm}
%\begin{algorithmic}
%\STATE $A \la$ indexy závodníků seřazené \textsc{MergeSort}em sestupně podle $s_i$
%\STATE $pmax \la 0$
%\STATE $smax \la 0$
%\FOR{$i \la 1 $ to $N$}
%\STATE $pmax \la pmax + p_{A[i]}$
%\STATE $smax \la \max(pmax+s_{A[i]},smax)$
%\ENDFOR
%\RETURN $smax$
%\end{algorithmic}
%\end{algorithm}
%
%\noindent
%\textsc{MergeSort} má časovou složitost $\O(N\log N)$, for cyklus se provede pouze $N$-krát.
%Celkově je tedy algoritmus v $\O(N\log N)$.

\bigskip

\noindent
Nyní chceme dokázat, že posloupnost (permutace) $S = \langle a_1,a_2,...,a_N \rangle$ indexů
závodníků získaná tímto algoritmem, tedy seřazená sestupně podle $s_i$, je optimální.

Definujme si funkci $v$, která pro danou permutaci indexů závodníků určí trvání závodu (říkejme mu cena),
$$ v(\langle a_1,a_2,\cdots,a_m \rangle) = \sum_{i=1}^m p_{a_i} + \max_{i=1}^m u(i)$$
$$ u(i) = s_{a_i} - \sum_{j=i+1}^m p_{a_j}$$
tedy celkové trvání, kdy budou závodníci plavat plus nejdelší čas, který nějaký závodník poběží od
chvíle, kdy všichni doplavali. Pomocná funkce $u(i)$ říká kolik času uběhlo mezi tím co doplaval
poslední závodník a mezi tím, co závodník $a_i$ doběhl do cíle. Její hodnota může být záporná,
nicméně alespoň pro index $m$ bude vždy kladná.

\clearpage
\zadani{2}{2}

\noindent
Předpokládejme existenci posloupnosti $T = \langle b_1,b_2,\cdots,b_N \rangle$, pro kterou platí $S
\neq T$ a jejíž cena je optimální. Protože $S \neq T$, existují indexy $b_k, b_{k+1}$ takové, že
$s_{b_k} < s_{b_{k+1}}$. Ukážeme, že prohodíme-li tyto dva prvky posloupnosti (kterou si označíme
$T'$), její cena se nezvýší.

Označme si $u'(i)$ hodnotu $u(i)$ na takto vzniklé posloupnosti. Zřejmě platí, že $u'(i) = u(i)$ pro
$i \notin \{k, k+1\}$ -- pro $i < k$ se změní pouze pořadí indexů větší než $i$, ne součet jejich
$p$; pro $i > k+1$ pak $p_{b_k}, p_{b_{k+1}}$ nejsou ani v sumě obsaženy. Dále platí:

$$
\begin{array}{rclclcl}
u(k)    & = & s_{b_k}     & - & p_{b_{k+1}} & - & \sum_{j=k+2}^N p_{b_j} \\
u(k+1)  & = & s_{b_{k+1}} &   &             & - & \sum_{j=k+2}^N p_{b_j} \\
u'(k)   & = & s_{b_k}     &   &             & - & \sum_{j=k+2}^N p_{b_j} \\
u'(k+1) & = & s_{b_{k+1}} & - & p_{b_k}     & - & \sum_{j=k+2}^N p_{b_j} \\
\end{array}
$$

\noindent
Protože $s_{b_k} < s_{b_{k+1}}$, vyplývá z těchto rovnic, že $u'(k) < u(k+1)$ a také, že $u'(k+1) <
u(k+1)$. Takže $\max u'(i)$ nebude větší než $\max u(i)$, cena se nezvýší.

\medskip
\noindent
Prohozením těchto dvou prvků tedy získáme jeden ze tří případů:

\begin{itemize}
\item Cena takto změněné posloupnosti je menší. To je spor s optimalitou $T$.
\item Cena zůstala stejná a $S = T'$. To znamená, že i $S$ je optimální.
\item Cena zůstala stejná a $S \neq T'$. Pak opět můžeme prohodit dva prvky, což děláme tak dlouho,
      dokud nedostaneme jeden z předchozích dvou případů (protože provádíme de facto BubbleSort,
      stane se tak v konečném počtu kroků).
\end{itemize}

\noindent
Permutace závodníků $S$ je tedy optimálním řešením problému.
\hfill$\square$

\clearpage
\zadani{2}{3}

\noindent
Problém má optimální substrukturu, kterou můžeme popsat funkcí $v$, která udává optimální cenu
rozdělení $st$ studentů do tří skupin o kapacitách $s_1, s_2$ resp. $s_3$. První případ nám říká, že
jakékoliv rozdělení nula studentů má hodnotu nula. Druhý případ ošetřuje možnost, kdy se nějáká
skupina dostane na zápornou kapacitu. Význam třetího případu je ten, že optimální rozdělení vznikne
umístěním studenta $st$ do některé ze skupin a rekurzivní umístění ostatních studentů do ostatních
skupin s odpovídajícím způsobem sníženou kapacitou; ze tří možností vybíráme maximum. Cenou
optimálního řešení celého problému je potom $v\left(N,\frac{N}{3},\frac{N}{3},\frac{N}{3}\right)$,
výsledné rozdělení je možné získat analogickým postupem jakým počítáme cenu.

%uglyhack
\newlength{\xlen}\settowidth{\xlen}{$\max \{$}
$$v(st,s_1,s_2,s_3) = \begin{cases}
0       & \text{pokud } st = 0 \\
-\infty & \text{pokud } s_1 < 0 \vee s_2 < 0 \vee s_3 < 0 \\
\max \{ P[st,1] + v(st-1,s_1-1,s_2,s_3), & \\
        \hspace{\xlen} P[st,2] + v(st-1,s_1,s_2-1,s_3), & \text{jinak}\\
        \hspace{\xlen} P[st,3] + v(st-1,s_1,s_2,s_3-1)\} &
\end{cases}$$

\noindent
Můžeme si všimnout, že $st = s_1 + s_2 + s_3$, takže můžeme tento parametr v algoritmu vynechat.
Algoritmus (na řádcích 3 až 25) na základě výše uvedené rovnosti vyplní trojrozměrnou tabulku $C$
hodnotami $v(s_1,s_2,s_3)\ \forall 0 \leq s_1,s_2,s_3 \leq \frac{N}{3}$.  Protože hodnota
$v(s_1,s_2,s_3)$ závisí na hodnotách $v$ s~o~jedna sníženými kapacitami skupin, probíhá vyplňování
tabulky v takovém pořadí, ve kterém součet $s_1 + s_2 + s_3$ neklesá.  V tabulce $G$ si algoritmus
uchovává skupinu, do které je přiřazen student $s_1+s_2+s_3$ při těchto kapacitách skupin. Tři cykly
for zajistí postupné projití skupin podle vzrůstajícího $s_1+s_2+s_3$, $C[s_1,s_2,s_3]$ se pak na
řádcích 9 až 21 vypočte pomocí hodnot, které byly vypočítané předtím; volba skupiny se uloží do pole
$G$.  Protože $d = s_1 + s_2 + s_3$, je alespoň jedna z kapacit skupin větší než nula, takže se
provede alespoň jeden if-blok a v $C[s_1,s_2,s_3]$ nikdy nezůstane hodnota $-\infty$.

Vnější cyklus je omezen $N$ iteracemi a oba vnější $\frac{N}{3}$ iteracemi, takže má tato část
algoritmu časovou složitost $\O(N^3)$.

Na řádcích 26 až 37 se pak z tabulky $G$ rekonstruuje přiřazení studentů do skupin, které se zapíše
do pole $Result$. $Result[st]$ obsahuje číslo skupiny, do které patří student $st$. Tělo while cyklu
se provede $N$-krát, protože součet $s_1+s_2+s_3$ konverguje k nule, složitost je tedy $\O(N)$.

Celková složitost algortimu je tím pádem $\O(N^3)$.

\clearpage
\zadani{2}{3}

\textsc{MakeSchedule}$(N,P)$
\begin{algorithmic}[1]
\STATE $C, G \la$ vytvoř trojrozměrné pole o velikosti $ N/3 \times N/3 \times N/3 $
\STATE $C[0,0,0] \la 0$
\FOR{$d \la 1$ to $N$}
\FOR{$s_1 \la 0$ to $\min(N/3,d)$}
\FOR{$s_2 \la 0$ to $\min(N/3,d-s_1)$}
\STATE $s_3 \la d-s_1-s_2$
\STATE $st \la s_1+s_2+s_3$
\IF{$s_3 \leq N/3$}
\STATE $C[s_1,s_2,s_3] \la -\infty$
\IF{$s_1 > 0 \wedge P[st,1] + C[s_1-1,s_2,s_3] > C[s_1,s_2,s_3]$}
\STATE $C[s_1,s_2,s_3] \la P[st,1] + C[s_1-1,s_2,s_3]$
\STATE $G[s_1,s_2,s_3] \la 1$
\ENDIF
\IF{$s_2 > 0 \wedge P[st,2] + C[s_1,s_2-1,s_3] > C[s_1,s_2,s_3]$}
\STATE $C[s_1,s_2,s_3] \la P[st,2] + C[s_1,s_2-1,s_3]$
\STATE $G[s_1,s_2,s_3] \la 2$
\ENDIF
\IF{$s_3 > 0 \wedge P[st,3] + C[s_1,s_2,s_3-1] > C[s_1,s_2,s_3]$}
\STATE $C[s_1,s_2,s_3] \la P[st,3] + C[s_1,s_2,s_3-1]$
\STATE $G[s_1,s_2,s_3] \la 3$
\ENDIF
\ENDIF
\ENDFOR
\ENDFOR
\ENDFOR

\STATE $s_1,s_2,s_3 \la N/3$
\STATE $Result \la$ vytvoř pole o délce $N$
\WHILE{$s_1+s_2+s_3 > 0$}
\STATE $Result[s_1+s_2+s_3] \la G[s_1,s_2,s_3]$
\IF{$G[s_1,s_2,s_3] = 1$}
\STATE $s_1 \la s_1 - 1$
\ELSIF{$G[s_1,s_2,s_3] = 2$}
\STATE $s_2 \la s_2 - 1$
\ELSE
\STATE $s_3 \la s_3 - 1$
\ENDIF
\ENDWHILE
\RETURN $Result$
\end{algorithmic}

\clearpage
\zadani{2}{4}

\noindent
Uvažme posloupnost ($5, 1, 1, 100, 10$) pro~počty operací na~počítači $A$, posloupnost
($1,1,10,1,1$) pro~počítač $B$. Algoritmus s~daným hladovým kritériem napočítá
plán ($A,P,B,P,A$), jehož cena je $25$. Plán ($A,A,A,A,A$) má však
cenu $117$, a to je spor s~optimalitou hladového kritéria.

\noindent
Navrhneme algoritmus založený na~principech dynamického programování. 
Nechť posloupnost $$A[i] \mid 1\leq i \leq n$$ označuje maximální počet operací, který je
možno provést za~$i$ minut, přičemž po~těchto $i$ minutách je úloha na~počítači~$A$.
Posloupnost $A'[i]$ označuje počítač, na~kterém byl výpočet v~minutě~$i$ (popř.
$P$ pro~přesun).
Symetricky pro~počítač $B$.

\noindent
Pak algoritmus:
\begin{algorithm}
\textsc{Scheduler}($(a_1,\dots,a_n),(b_1,\dots,b_n)$)
\begin{algorithmic}[1]
\STATE $A[1] \la a_1$; $B[1] \la b_1$
\FOR {$i \la 2~to~n$}
\STATE $A[i] \la$ \textsc{Max}($A[i-1]+a_i$, $B[i-1]$)
\STATE $A'[i] \la$ $A$ pokud $A[i-1]+a_i \geq B[i-1]$, $P$ jinak
\STATE $B[i] \la$ \textsc{Max}($B[i-1]+b_i$, $A[i-1]$)
\STATE $B'[i] \la$ $B$ pokud $B[i-1]+b_i \geq A[i-1]$, $P$ jinak
\ENDFOR
\IF {$A[n] > B[n]$}
\RETURN zrekonstruuj plán z $A',B'$, přičemž $p_n = A$
\ELSE
\RETURN zrekonstruuj plán z $A',B'$, přičemž $p_n = B$
\ENDIF
\end{algorithmic}
\end{algorithm}

\noindent
Rekonstrukce plánu:
\begin{equation*}
p_{i-1} = \left\{
	\begin{array}{cl}
	A'[i-1] & \text{pokud $p_{i} = A~\vee (p_i=P \wedge p_{i+1}=B)$}\\
	B'[i-1] & \text{pokud $p_{i} = B~\vee (p_i=P \wedge p_{i+1}=A)$}\\
	\bot & \text{jinak}
	\end{array} \right.
\end{equation*}

%\begin{algoritmic}
%\FOR {$i \la n~downto~1$}
%\IF {$p_i = A$}
%\STATE $p_{i-1} \la A'[i-1]$
%\ELSIF {$p_i = B$}
%\STATE $p_{i-1} \la B'[i-1]$
%\ELSIF {$p_i = P$}
%\STATE pokud $p_{i+1} = A$, tak 
%\ENDIF
%\ENDFOR
%\RETURN (p_1, \dots, p_n)
%\end{algoritmic}

\clearpage
\zadani{2}{4}
\noindent
Analýza algoritmu \textsc{Scheduler}:\\
\textbf{Složitost}\\
Uvnitř \textit{for} cyklu z~řádku $2$ jsou pouze operace se složitostí~$\in~\mathcal{O}(1)$, tedy cena tohoto \textit{for} cyklu je~$\in~\mathcal{O}(n)$.
Cena rekonstrukce plánu je~$\in
\mathcal{O}(n)$. Tedy složitost algorimtu \textsc{Scheduler}
je~$\in~\mathcal{O}(n)$.\\
\textbf{Korektnost}\\
\renewcommand{\labelenumi}{\textbf{\alph{enumi})}}
Nejdříve dokážeme tvrzení
\begin{quotation}
$A[i] \mid 1\leq i \leq n$ je maximální počet operací, který je
možno provést za~$i$~minut, přičemž po~těchto $i$ minutách je úloha na~počítači
$A$.
\end{quotation}
Důkaz indukcí vzhledem k parametru $i$.
\begin{description}
\item [Báze]{$i=1$} -- triviální -- $a_1 \geq $ počet operací při přesunu
\item [Krok]{$i=n$} -- z~indukčního předpokladu známe $A[i-1]$ a $B[i-1]$, které
jsou maximální. Existují $2$
možnosti jak skončit výpočet na~počítači $A$. Buď se v~$i$-té minutě přesunujeme
z~počítače $B$, nebo v~$i$-té minutě počítáme na~počítači $A$. Pokud tedy napočítáme
výsledné počty operací, nutně větší z~nich musí být součástí maximálního plánu
pro~$i$ minut se skončením na~počítači $A$.
\end{description}
Symetricky pro~počítač $B$ a posloupnost $B[i]$.

\noindent
Po~$n$ minutách jeden z~optimálních výpočtů skončí na~počítači $A$, nebo $B$. Pokud by
se výpočet přesouval, jistě by se nevykonalo více operací než bez přesunu.
Větší z~čísel $A[n], B[n]$ tedy udává maximální počet operací po~$n$ minutách a
zároveň známe počítač, na~kterém tento výpočet skončil. 

\noindent
Nyní dokážeme korektnost zrekonstruování plánu indukcí dle $n-i$.
BÚNO předpokládejme, že výpočet skončil na~počítači $A$, tedy $p_n=A$.
\begin{description}
\item [Báze]{$i=0$} -- triviální -- $p_n=A$.
\item [Krok]{$1\leq i \leq n$} -- triviální pokud $p_{n-i+1} = A~\vee p_{n-i+1}
= B$. Pokud $p_{n-i+1} = P$, tak se výpočet v~$n-i+1.$ minutě přesouval, tedy
předchozí minutu musel být na~opačném stroji.
\end{description}
Symetricky pro~$p_n=B$.\\
\textbf{Konvergence} -- zřejmé -- pouze cykly s~přesně definovaným počtem opakování.
\hfill$\square$

\clearpage
\zadani{2}{5}

\noindent
Algoritmus ověří, zda je aktuálně zpracovávaný vrchol extremální a pokud není, znamená to, že kořen
alespoň jednoho podstromu má menší hodnotu. Na tento podstrom se algoritmus rekurzivně zavolá (pokud
mají menší hodnotu vrcholy obou podstromů, zavolá se na levý). Takto vždy dojde k extremálnímu
vrcholu -- viz níže.

\begin{algorithm}
\textsc{FindExtremal}($node$)
\begin{algorithmic}
\IF{$\textsc{IsLeaf}(node)$}
\RETURN $node$
\ELSIF{$\textsc{Value}(node.left) < \textsc{Value}(node)$}
\RETURN \textsc{FindExtremal}($node.left$)
\ELSIF{$\textsc{Value}(node.right) < \textsc{Value}(node)$}
\RETURN \textsc{FindExtremal}($node.right$)
\ELSE
\RETURN $node$
\ENDIF
\end{algorithmic}
\end{algorithm}

Invariantem funkce \textsc{FindExtremal} je, že rodič aktuálně zpracovávaného vrcholu je větší než
on sám (nebo je zpracovávaný rodič kořen). V kořeni triviálně platí a rekurzivní volání na podstrom
s menší hodnotou kořene jej zachovávají. Pokud platí invariant a oba podstromy mají kořen s větší
hodnotou, je vrchol extremální. Podobně, pokud platí invariant a zpracovávaný vrchol je listem, je
tento list menší než všechny vrcholy a je tedy rovněž extremální.

Z toho, že se algoritmus rekurzivně volá na jeden z podstromů, kterýžto má o jedna menší výšku,
vyplývají následující fakta:

\begin{itemize}
\item Algoritmus je korektní, protože vždy dojde buď do listu nebo předtím narazí na vnitřní
extremální vrchol.
\item Algoritmus je konvergentní.
\item Protože se v každé instanci volá \textsc{Value} nejvýše čtyřikrát\footnote{počet těchto volání
by bylo možné snížit na dvě} a těchto instancí je nejvýše tolik, kolik je výška stromu, tedy $\log
n$, je počet volání $\O(\log n)$.
\end{itemize}

\hfill$\square$

\end{document}
