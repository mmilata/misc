\documentclass[12pt]{article}
\usepackage[czech]{babel}
\usepackage[utf8]{inputenc}
\usepackage{amsmath,amssymb}
\usepackage{enumerate}
%\usepackage{hyperref}
\usepackage{algorithmic}
\usepackage{algorithm}
%\usepackage{tikz}


\pagestyle{plain}

\topmargin -5mm
\headsep 0mm
\headheight 3mm
\textheight 235mm
\textwidth 165mm
\oddsidemargin 0mm
\evensidemargin 0mm
\footskip 10mm

\newcommand{\eps}{\varepsilon}
\newcommand{\coL}{co\mbox{$-$}L}
\newcommand{\move}{\rightarrow}
\newcommand{\la}{\leftarrow}
\newcommand{\var}[1]{\textit{#1}}

\renewcommand{\O}{\mathcal{O}}

\newcommand{\zadani}[2]{
{\large
\noindent {\bf IB108 \hfill{} Sada #1, Příklad #2 \\[-4mm]}
\noindent\hrule
\vspace{2mm}
\noindent Vypracovali:\hfill{}Tomáš Krajča (255676), Martin Milata (256615)
\vspace{3mm}
\hrule
\bigskip\bigskip}
}


\begin{document}

\zadani{2}{1}

\noindent
%Nechť $d:student \times student \move \{yes, no\}$ je dotaz do databáze.
Nechť pole \var{students}[1..N] reprezentuje soubor studentů.
Nechť výsledek funkce \textsc{Schoolmates} je kódovaný takto:
\begin{equation*}
\textsc{Schoolmates}(1,N) = \left\{
	\begin{array}{rl}
	0 & \text{\var{students}[1..N] neobsahuje alespoň $\left\lceil \frac{N}{2}
	\right\rceil$ spolužáků}\\
	1 & \text{\var{students}[1..t] jsou spolužáci $\mid t \geq \left\lceil \frac{N}{2}
	\right\rceil$}\\
	2 & \text{\var{students}[1..$\frac{N}{2}
	$], \var{students}[$(\frac{N}{2}
	+1)$..N] jsou
	spolužáci $\mid N$ sudé}
	\end{array} \right.
\end{equation*}
Hodnota $2$ značí, že \var{students}[1..N] obsahuje $2$ různé skupiny
spolužáků.\\
Tedy hodnota $1$ nebo $2$ reprezentuje odpověď \textbf{ano}, zatímco hodnota $0$
reprezentuje odpověď \textbf{ne}.\\
Algoritmus využívá toho, že pokud známe výsledek pro~$2$ podobně dlouhé
podposloupnosti, lehce určíme výsledek i pro~jejich sloučení (máme $3$ různé
možnosti). Pro~zjednodušení řadíme v~poli studenty tak, aby vždy v~první části byli
spolužáci (pro~výsledky $1$, $2$).

\bigskip
%\vskip 1cm
\noindent
Analýza algoritmu \textsc{Schoolmates}:\\
\textbf{Složitost}\footnote{Počet dotazů do databáze by bylo možné optimalizovat.}
\begin{itemize}
\item \textbf{if} na řádku $4$ obsahuje $2\cdot\left\lceil\frac{N}{2}\right\rceil \leq
N+1$ dotazů.
\item \textbf{if} na řádku $16$ obsahuje $\left\lceil\frac{N}{2}\right\rceil +
\left\lfloor\frac{N}{4}\right\rfloor \leq N$ dotazů. Víme, že v jednom z polí je
alespoň $\left\lceil\frac{N}{4}\right\rceil$ spolužáků za sebou. Pro zjištění
hodnoty $x$ stačí tedy $\left\lfloor\frac{N}{4}\right\rfloor$ dotazů.
\item \textbf{if} na řádku $25$ obsahuje $0$ dotazů.
\end{itemize}
Tedy rekurentně $T(1) = 0, T(2) = 0, T(N) \leq 2\cdot T(N/2) + (N+1)$ dotazů.\\
Dle \textit{Master Theorem} je tedy počet dotazů $\in \mathcal{O}(N\log N)$.\\
\textbf{Korektnost} -- Důkaz indukcí k počtu studentů (délce pole $N = R - L + 1$).

\renewcommand{\labelenumi}{\textbf{\alph{enumi})}}
\begin{itemize}
\item \textbf{Báze} -- počet studentů $N=1$, $N=2$ -- triviální -- žák je sám sobě spolužákem
\item \textbf{Krok} -- z indukčního předpokladu víme, že \var{Schoolm1} a
\var{Schoolm2} mají správný výsledek pro počet studentů $ < N$. Nyní existují
následující $3$ možnosti:\begin{enumerate}
\item \var{Schoolm1} $ = 2~\vee $ \var{Schoolm2} $ = 2$ -- v jednom z polí jsou
$2$ skupiny spolužáků (každá skupina má $\frac{N}{4}$ spolužáků). BÚNO je to
\var{Schoolm1}. Pokud \var{Schoolm2} obsahuje $2$ totožné skupiny spolužáků,
bude sloučené pole obsahovat $2$ skupiny po $\frac{N}{2}$ spolužácích. Jinak pokud
\var{Schoolm2} obsahuje alespoň $\frac{N}{4}$ spolužáků s jednou ze skupin v
\var{Schoolm1}, bude i sloučené pole obsahovat alespoň $\frac{N}{2}$ těchto
spolužáků. Jinak sloučené pole nebude obsahovat alespoň $\frac{N}{2}$ spolužáků.
\item \var{Schoolm1} $ = 1~\vee $ \var{Schoolm2} $ = 1$ -- jedno z polí obsahuje
alespoň $\left\lceil\frac{N}{4}\right\rceil$ spolužáků. BÚNO je to
\var{Schoolm1} a těchto spolužáků je \var{x}. Pokud najdeme v \var{Schoolm2}
alespoň $\left\lceil\frac{N}{2}\right\rceil-x$ stejných spolužáků, bude i sloučené pole obsahovat alespoň
$\left\lceil\frac{N}{2}\right\rceil$ spolužáků, jinak ne.

\clearpage
\zadani{1}{1}
\item \var{Schoolm1} $ = 0~\wedge $ \var{Schoolm2} $ = 0$ -- v ani jednom z obou
 polí není alespoň $\left\lceil\frac{N}{4}\right\rceil$ spolužáků, ve sloučeném
 poli tedy může být maximálně $\left\lceil\frac{N}{2}\right\rceil-1$ spolužáků.
\end{enumerate}
\end{itemize}
\textbf{Konvergence} -- při rekurzivním volání funkce \textsc{Schoolmates} na
řádku $3$ vždy
snížíme velikost pole, na které je volána. Zřejmě v konečném počtu kroků musí
velikost pole klesnout pod hodnotu $3$. 
\hfill$\square$

\clearpage
\zadani{2}{2}
\begin{algorithm}
\textsc{Schoolmates}($L,R$)
\begin{algorithmic}[1]
\STATE $N \la R-L+1$
\IF {$N > 2$}
\STATE $Schoolm1 \gets$ \textsc{Schoolmates}($L,\left\lfloor \frac{R}{2}
\right\rfloor$); $Schoolm2 \gets$ \textsc{Schoolmates}($\left\lfloor \frac{R}{2} \right\rfloor+1,R$)
\IF{$Schoolm1 = 2 \vee Schoolm2 = 2$}
\STATE Určíme tu část pole, která vrátila výsledek $2$. BÚNO je to
$Schoolm1$.
\STATE Postupně porovnáváme všechny prvky z části pole $Schoolm2$ s vybraným
žákem z první a poté druhé části pole $Schoolm1$.
\IF {najdeme dalších $\frac{N}{4}$ spolužáků první části $\wedge$ dalších
$\frac{N}{4}$ spolužáků druhé části $\mid N$ sudé}
\STATE přeskládáme studenty tak, aby první polovina spojeného pole
byli spolužáci
\RETURN $2$
\ELSIF {najdeme dalších alespoň $\left\lceil\frac{N}{4}\right\rceil$
spolužáků}
\STATE přeskládáme studenty tak, aby první část spojeného pole tvořili tito spolužáci
\RETURN $1$
\ELSE
\RETURN $0$
\ENDIF
\ELSIF{$Schoolm1 = 1 \vee Schoolm2 = 1$}
\STATE Určíme tu část pole, která vrátila výsledek $1$ a počet spolužáků v její
první části. BÚNO je to
$Schoolm1$ a $x$.
\STATE Postupně porovnáme všechny studenty z části pole $Schoolm2$ s
vybraným žákem z první části pole $Schoolm2$
\IF{najdeme dalších alespoň $y$ spolužáků $\mid y+x\geq
\left\lceil\frac{N}{2}\right\rceil$}
\STATE přeskládáme studenty tak, aby první část spojeného pole tvořili tito spolužáci
\RETURN $1$
\ELSE
\RETURN $0$
\ENDIF
\ELSIF{$Schoolm1 = 0 \wedge Schoolm2 = 0$}
\RETURN $0$
\ENDIF
\ELSE
\RETURN $2$
\ENDIF
\end{algorithmic}
\end{algorithm}



\clearpage
\zadani{2}{2}

% TODO:
% - odstranit ten algoritmus? je docela zbytecny ...
% - vyresit kolize v namespacu promennych

\noindent
Zřejmě nezáleží na čase, kdy závodník běží a kdy jede na kole, záleží pouze na jejich součtu.
Označme si proto $s_i = k_i + b_i$.  Problém má optimální substrukturu, kterou můžeme
charakterizovat funkcí pro výpočet minimální délky závodu:
$$t(\emptyset) = 0$$
$$t(I) = 
\min_{i \in I} \left(
		p_i + \max \left\{
				s_i, t(I\smallsetminus\left\{i\right\})
		\right\}
\right)$$

Ukážeme, že při využití hladového kritéria podle kterého vybíráme jako prvního závodníka s největším
$s_i$ získáme optimální výsledek. Tedy že $t(I) = p_m + max(s_m, t(I \smallsetminus \{m\}))$, kde $m
\in I$ je takové, že $s_m$ je maximální.

Algoritmus, který do pole $A$ zapíše optimální pořadí závodníků a spočítá čas, kdy doběhne poslední,
může vypadat následovně.

\begin{algorithm}
\begin{algorithmic}
\STATE $A \la$ indexy závodníků seřazené \textsc{MergeSort}em sestupně podle $s_i$
\STATE $pmax \la 0$
\STATE $smax \la 0$
\FOR{$i \la 1 $ to $N$}
\STATE $pmax \la pmax + p_{A[i]}$
\STATE $smax \la \max(pmax+s_{A[i]},smax)$
\ENDFOR
\RETURN $smax$
\end{algorithmic}
\end{algorithm}

\noindent
\textsc{MergeSort} má časovou složitost $\O(N\log N)$, for cyklus se provede pouze $N$-krát.
Celkově je tedy algoritmus v $\O(N\log N)$.

\bigskip

\noindent
Nyní chceme dokázat, že posloupnost $S = \langle i_1,i_2,...,i_N \rangle$ indexů závodníků získaná tímto
algoritmem, tedy seřazená sestupně podle $s_i$ je optimální.

Definujme si funkci, která pro danou permutaci indexů závodníků určí trvání závodu (říkejme mu cena),
%$$ v(a_1,a_2,\cdots,a_m) = \sum_{i=1}^N p_{a_i} + \max_{i=1}^N \left(s_{a_i} - \Sigma_{j=i}^N p_{a_j}\right)$$
$$ v(\langle a_1,a_2,\cdots,a_m \rangle) = \sum_{i=1}^m p_{a_i} + \max_{i=1}^m u(i)$$
$$ u(i) = s_{a_i} - \sum_{j=i+1}^m p_i$$
tedy celkové trvání, kdy budou závodníci plavat plus nejdelší čas, který nějáký závodník poběží od
chvíle, kdy všichni doplavali. Funkce $u(i)$ říká kolik času uběhlo mezi tím co doplaval poslední
závodník a mezi tím, co závodník $a_i$ doběhl do cíle. Její hodnota může být záporná, nicméně
alespoň pro index $m$ bude vždy kladná.

\clearpage
\zadani{2}{2}

\noindent
Předpokládejme existenci posloupnosti $T = \langle j_1,j_2,...,j_N \rangle$, pro kterou platí $S
\neq T$ a jejíž cena je optimální. Protože $S \neq T$, existují indexy $j_k, j_{k+1}$ takové, že
$s_{j_k} < s_{j_{k+1}}$. Ukážeme, že prohodíme-li tyto dva prvky posloupnosti (kterou si označíme
$T'$), její cena se nezvýší.

Označme si $u'(i)$ hodnotu $u(i)$ na takto vzniklé posloupnosti. Zřejmě platí, že $u'(i) = u(i)$ pro
$i \notin \{k, k+1\}$ -- pro $i < k$ se změní pouze pořadí indexů větší než $i$, ne součet jejich
$p$; pro $i > k+1$ pak $p_{j_k}, p_{j_{k+1}}$ nejsou ani v sumě obsaženy. Dále platí:

$$
\begin{array}{rclclcl}
u(k)    & = & s_{j_k}     & - & p_{j_{k+1}} & - & \Sigma_{j=k+2}^N p_j \\
u(k+1)  & = & s_{j_{k+1}} &     &               & - & \Sigma_{j=k+2}^N p_j \\
u'(k)   & = & s_{j_k}     &     &               & - & \Sigma_{j=k+2}^N p_j \\
u'(k+1) & = & s_{j_{k+1}} & - & p_k         & - & \Sigma_{j=k+2}^N p_j \\
\end{array}
$$

\noindent
Protože $s_{j_k} < s_{j_{k+1}}$, vyplývá z těchto rovnic, že $u'(k) < u(k+1)$ a také, že $u'(k+1) <
u(k+1)$. Takže $\max u'(i)$ nebude větší než $\max u(i)$, cena se nezvýší.

\medskip
\noindent
Získáváme tedy jeden ze tří případů:
\begin{itemize}
\item Cena takto změněné posloupnosti je menší. To je spor s optimalitou $T$.
\item Cena zůstala stejná a $S = T'$. To znamená, že i $S$ je optimální.
\item Cena zůstala stejná a $S \neq T'$. Pak opět můžeme prohodit dva prvky, což děláme tak dlouho,
      dokud nedostaneme jeden z předchozích dvou případů.
\end{itemize}

%Zavedeme si funkci $v$, která nám pro
%danou posloupnost závodníků určí celkové trvání závodu a říkejme tomuto trvání cena.
%
%$$ v(\langle \rangle) = 0 $$
%$$ v(\langle a_1,a_2,...,a_m \rangle) = p_{a_1} + \max \{s_{a_1}, v(\langle a_2,...,a_m \rangle)\} $$
%
%\noindent
%Předpokládejme existenci posloupnosti $T = \langle j_1,j_2,...,j_N \rangle$, pro kterou platí $S
%\neq T$ a která je optimální. %Podle definice $v$ existuje $k$ takové, že
%%$$ v(T) = p_{j_1} + p_{j_2} + ... + p_{j_k} + s_{j_k}. $$


\clearpage
\zadani{2}{3}



\clearpage
\zadani{2}{4}



\clearpage
\zadani{2}{5}

\noindent
Algoritmus ověří, zda je aktuálě zpracovávaný vrchol extremální a pokud není, znamená to, že kořen
alespoň jednoho podstromu má menší hodnotu. Na tento podstrom se algoritmus rekurzivně zavolá (pokud
mají menší hodnotu vrcholy obou podstromů, zavolá se na levý). Takto vždy dojde k extremálnímu
vrcholu -- viz níže.

%\begin{algorithm}
%\begin{algorithmic}
%\STATE $node \la root$
%\WHILE{not $\textsc{Leaf}(node)$}
%\IF{$\textsc{Value}(node.left) < \textsc{Value}(node)$}
%\STATE $node \la node.left$
%\ELSIF{$\textsc{Value}(node.right) < \textsc{Value}(node)$}
%\STATE $node \la node.right$
%\ELSE
%\RETURN $node$
%\ENDIF
%\ENDWHILE
%\RETURN $node$
%\end{algorithmic}
%\end{algorithm}

\begin{algorithm}
\textsc{FindExtremal}($node$)
\begin{algorithmic}
\IF{$\textsc{IsLeaf}(node)$}
\RETURN $node$
\ELSIF{$\textsc{Value}(node.left) < \textsc{Value}(node)$}
\RETURN \textsc{FindExtremal}($node.left$)
\ELSIF{$\textsc{Value}(node.right) < \textsc{Value}(node)$}
\RETURN \textsc{FindExtremal}($node.right$)
\ELSE
\RETURN $node$
\ENDIF
\end{algorithmic}
\end{algorithm}

Invariantem funkce \textsc{FindExtremal} je, že rodič aktuálně zpracovávaného vrcholu je větší než
on sám (nebo je zpracovávaný rodič kořen). V kořeni triviálně platí a rekurzivní volání na podstrom
s menší hodnotou kořene jej zachovávají. Pokud platí invariant a oba podstromy mají kořen s větší
hodnotou, je vrchol extremální. Podobně, pokud platí invariant a zpracovávaný vrchol je listem, je
tento list menší než všechny vrcholy a je tedy rovněž extremální.

Z toho, že se algoritmus rekurzivně volá na jeden z podstromů, kterýžto má o jedna menší výšku,
vyplývají následující fakta:

\begin{itemize}
\item Algoritmus je korektní, protože vždy dojde buď do listu nebo předtím narazí na vnitřní
extremální vrchol.
\item Algoritmus je konvergentní.
\item Protože se v každé instanci volá \textsc{Value} nejvýše čtyřikrát\footnote{počet těchto volání
by bylo možné snížit na dvě} a těchto instancí je nejvýše tolik, kolik je výška stromu, tedy $\log
n$, je počet volání $\O(\log n)$.
\end{itemize}

\hfill$\square$

\end{document}
