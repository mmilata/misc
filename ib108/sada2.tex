\documentclass[12pt]{article}
\usepackage[czech]{babel}
\usepackage[utf8]{inputenc}
\usepackage{amsmath,amssymb}
\usepackage{enumerate}
%\usepackage{hyperref}
\usepackage{algorithmic}
\usepackage{algorithm}
%\usepackage{tikz}


\pagestyle{plain}

\topmargin -5mm
\headsep 0mm
\headheight 3mm
\textheight 235mm
\textwidth 165mm
\oddsidemargin 0mm
\evensidemargin 0mm
\footskip 10mm

\newcommand{\eps}{\varepsilon}
\newcommand{\coL}{co\mbox{$-$}L}
\newcommand{\move}{\rightarrow}
\newcommand{\la}{\leftarrow}
\newcommand{\var}[1]{\textit{#1}}

\renewcommand{\O}{\mathcal{O}}

\newcommand{\zadani}[2]{
{\large
\noindent {\bf IB108 \hfill{} Sada #1, Příklad #2 \\[-4mm]}
\noindent\hrule
\vspace{2mm}
\noindent Vypracovali:\hfill{}Tomáš Krajča (255676), Martin Milata (256615)
\vspace{3mm}
\hrule
\bigskip\bigskip}
}


\begin{document}

\zadani{2}{1}

\noindent
%Nechť $d:student \times student \move \{yes, no\}$ je dotaz do databáze.
Nechť pole \var{students}[1..N] reprezentuje soubor studentů.
Nechť výsledek funkce \textsc{Schoolmates} je kódovaný takto:
\begin{equation*}
\textsc{Schoolmates}(1,N) = \left\{
	\begin{array}{rl}
	0 & \text{\var{students}[1..N] neobsahuje alespoň $\left\lceil \frac{N}{2}
	\right\rceil$ spolužáků}\\
	1 & \text{\var{students}[1..t] jsou spolužáci $\mid t \geq \left\lceil \frac{N}{2}
	\right\rceil$}\\
	2 & \text{\var{students}[1..$\frac{N}{2}
	$], \var{students}[$(\frac{N}{2}
	+1)$..N] jsou
	spolužáci $\mid N$ sudé}
	\end{array} \right.
\end{equation*}
Hodnota $2$ značí, že \var{students}[1..N] obsahuje $2$ různé skupiny
spolužáků.\\
Tedy hodnota $1$ nebo $2$ reprezentuje odpověď \textbf{ano}, zatímco hodnota $0$
reprezentuje odpověď \textbf{ne}.\\
Algoritmus využívá toho, že pokud známe výsledek pro~$2$ podobně dlouhé
podposloupnosti, lehce určíme výsledek i pro~jejich sloučení (máme $3$ různé
možnosti). Pro~zjednodušení řadíme v~poli studenty tak, aby vždy v~první části byli
spolužáci (pro~výsledky $1$, $2$).

\bigskip
%\vskip 1cm
\noindent
Analýza algoritmu \textsc{Schoolmates}:\\
\textbf{Složitost}\footnote{Počet dotazů do databáze by bylo možné optimalizovat.}
\begin{itemize}
\item \textbf{if} na řádku $4$ obsahuje $2\cdot\left\lceil\frac{N}{2}\right\rceil \leq
N+1$ dotazů.
\item \textbf{if} na řádku $16$ obsahuje $\left\lceil\frac{N}{2}\right\rceil +
\left\lfloor\frac{N}{4}\right\rfloor \leq N$ dotazů. Víme, že v jednom z polí je
alespoň $\left\lceil\frac{N}{4}\right\rceil$ spolužáků za sebou. Pro zjištění
hodnoty $x$ stačí tedy $\left\lfloor\frac{N}{4}\right\rfloor$ dotazů.
\item \textbf{if} na řádku $25$ obsahuje $0$ dotazů.
\end{itemize}
Tedy rekurentně $T(1) = 0, T(2) = 0, T(N) \leq 2\cdot T(N/2) + (N+1)$ dotazů.\\
Dle \textit{Master Theorem} je tedy počet dotazů $\in \mathcal{O}(N\log N)$.\\
\textbf{Korektnost} -- Důkaz indukcí k počtu studentů (délce pole $N = R - L + 1$).

\renewcommand{\labelenumi}{\textbf{\alph{enumi})}}
\begin{itemize}
\item \textbf{Báze} -- počet studentů $N=1$, $N=2$ -- triviální -- žák je sám sobě spolužákem
\item \textbf{Krok} -- z indukčního předpokladu víme, že \var{Schoolm1} a
\var{Schoolm2} mají správný výsledek pro počet studentů $ < N$. Nyní existují
následující $3$ možnosti:\begin{enumerate}
\item \var{Schoolm1} $ = 2~\vee $ \var{Schoolm2} $ = 2$ -- v jednom z polí jsou
$2$ skupiny spolužáků (každá skupina má $\frac{N}{4}$ spolužáků). BÚNO je to
\var{Schoolm1}. Pokud \var{Schoolm2} obsahuje $2$ totožné skupiny spolužáků,
bude sloučené pole obsahovat $2$ skupiny po $\frac{N}{2}$ spolužácích. Jinak pokud
\var{Schoolm2} obsahuje alespoň $\frac{N}{4}$ spolužáků s jednou ze skupin v
\var{Schoolm1}, bude i sloučené pole obsahovat alespoň $\frac{N}{2}$ těchto
spolužáků. Jinak sloučené pole nebude obsahovat alespoň $\frac{N}{2}$ spolužáků.
\item \var{Schoolm1} $ = 1~\vee $ \var{Schoolm2} $ = 1$ -- jedno z polí obsahuje
alespoň $\left\lceil\frac{N}{4}\right\rceil$ spolužáků. BÚNO je to
\var{Schoolm1} a těchto spolužáků je \var{x}. Pokud najdeme v \var{Schoolm2}
alespoň $\left\lceil\frac{N}{2}\right\rceil-x$ stejných spolužáků, bude i sloučené pole obsahovat alespoň
$\left\lceil\frac{N}{2}\right\rceil$ spolužáků, jinak ne.

\clearpage
\zadani{1}{1}
\item \var{Schoolm1} $ = 0~\wedge $ \var{Schoolm2} $ = 0$ -- v ani jednom z obou
 polí není alespoň $\left\lceil\frac{N}{4}\right\rceil$ spolužáků, ve sloučeném
 poli tedy může být maximálně $\left\lceil\frac{N}{2}\right\rceil-1$ spolužáků.
\end{enumerate}
\end{itemize}
\textbf{Konvergence} -- při rekurzivním volání funkce \textsc{Schoolmates} na
řádku $3$ vždy
snížíme velikost pole, na které je volána. Zřejmě v konečném počtu kroků musí
velikost pole klesnout pod hodnotu $3$. 
\hfill$\square$

\clearpage
\zadani{2}{2}
\begin{algorithm}
\textsc{Schoolmates}($L,R$)
\begin{algorithmic}[1]
\STATE $N \la R-L+1$
\IF {$N > 2$}
\STATE $Schoolm1 \gets$ \textsc{Schoolmates}($L,\left\lfloor \frac{R}{2}
\right\rfloor$); $Schoolm2 \gets$ \textsc{Schoolmates}($\left\lfloor \frac{R}{2} \right\rfloor+1,R$)
\IF{$Schoolm1 = 2 \vee Schoolm2 = 2$}
\STATE Určíme tu část pole, která vrátila výsledek $2$. BÚNO je to
$Schoolm1$.
\STATE Postupně porovnáváme všechny prvky z části pole $Schoolm2$ s vybraným
žákem z první a poté druhé části pole $Schoolm1$.
\IF {najdeme dalších $\frac{N}{4}$ spolužáků první části $\wedge$ dalších
$\frac{N}{4}$ spolužáků druhé části $\mid N$ sudé}
\STATE přeskládáme studenty tak, aby první polovina spojeného pole
byli spolužáci
\RETURN $2$
\ELSIF {najdeme dalších alespoň $\left\lceil\frac{N}{4}\right\rceil$
spolužáků}
\STATE přeskládáme studenty tak, aby první část spojeného pole tvořili tito spolužáci
\RETURN $1$
\ELSE
\RETURN $0$
\ENDIF
\ELSIF{$Schoolm1 = 1 \vee Schoolm2 = 1$}
\STATE Určíme tu část pole, která vrátila výsledek $1$ a počet spolužáků v její
první části. BÚNO je to
$Schoolm1$ a $x$.
\STATE Postupně porovnáme všechny studenty z části pole $Schoolm2$ s
vybraným žákem z první části pole $Schoolm2$
\IF{najdeme dalších alespoň $y$ spolužáků $\mid y+x\geq
\left\lceil\frac{N}{2}\right\rceil$}
\STATE přeskládáme studenty tak, aby první část spojeného pole tvořili tito spolužáci
\RETURN $1$
\ELSE
\RETURN $0$
\ENDIF
\ELSIF{$Schoolm1 = 0 \wedge Schoolm2 = 0$}
\RETURN $0$
\ENDIF
\ELSE
\RETURN $2$
\ENDIF
\end{algorithmic}
\end{algorithm}



\clearpage
\zadani{2}{2}

% TODO:
% - odstranit ten algoritmus? je docela zbytecny ...
% - vyresit kolize v namespacu promennych

\noindent
Zřejmě nezáleží na čase, kdy závodník běží a kdy jede na kole, záleží pouze na jejich součtu.
Označme si proto $s_i = k_i + b_i$.  Problém má optimální substrukturu, kterou můžeme
charakterizovat funkcí pro výpočet minimální délky závodu ($I$ je množina indexů označujících
účastníky závodu):
$$t(\emptyset) = 0$$
$$t(I) = 
\min_{i \in I} \left(
		p_i + \max \left\{
				s_i, t(I\smallsetminus\left\{i\right\})
		\right\}
\right)$$

Ukážeme, že při využití hladového kritéria podle kterého vybíráme nejdříve závodníky s největším
$s_i$ získáme optimální výsledek. Tedy že $t(I) = p_m + max(s_m, t(I \smallsetminus \{m\}))$, kde $m
\in I$ je takové, že $s_m$ je maximální.

Algoritmus, který do pole $A$ zapíše optimální pořadí závodníků a spočítá čas, kdy doběhne poslední,
může vypadat následovně.

\begin{algorithm}
\begin{algorithmic}
\STATE $A \la$ indexy závodníků seřazené \textsc{MergeSort}em sestupně podle $s_i$
\STATE $pmax \la 0$
\STATE $smax \la 0$
\FOR{$i \la 1 $ to $N$}
\STATE $pmax \la pmax + p_{A[i]}$
\STATE $smax \la \max(pmax+s_{A[i]},smax)$
\ENDFOR
\RETURN $smax$
\end{algorithmic}
\end{algorithm}

\noindent
\textsc{MergeSort} má časovou složitost $\O(N\log N)$, for cyklus se provede pouze $N$-krát.
Celkově je tedy algoritmus v $\O(N\log N)$.

\bigskip

\noindent
Nyní chceme dokázat, že posloupnost (permutace) $S = \langle a_1,a_2,...,a_N \rangle$ indexů
závodníků získaná tímto algoritmem, tedy seřazená sestupně podle $s_i$, je optimální.

Definujme si funkci $v$, která pro danou permutaci indexů závodníků určí trvání závodu (říkejme mu cena),
%$$ v(\langle a_1,a_2,\cdots,a_m \rangle) = \sum_{i=1}^m p_{a_i} + \max_{i=1}^m u(i)$$
%$$ u(i) = s_{a_i} - \sum_{j=i+1}^m p_i$$
$$ v(\langle a_1,a_2,\cdots,a_m \rangle) = \sum_{i=1}^m p_{a_i} + \max_{i=1}^m u(i)$$
$$ u(i) = s_{a_i} - \sum_{j=i+1}^m p_{a_j}$$
tedy celkové trvání, kdy budou závodníci plavat plus nejdelší čas, který nějáký závodník poběží od
chvíle, kdy všichni doplavali. Pomocná funkce $u(i)$ říká kolik času uběhlo mezi tím co doplaval
poslední závodník a mezi tím, co závodník $a_i$ doběhl do cíle. Její hodnota může být záporná,
nicméně alespoň pro index $m$ bude vždy kladná.

\clearpage
\zadani{2}{2}

\noindent
Předpokládejme existenci posloupnosti $T = \langle b_1,b_2,\cdots,b_N \rangle$, pro kterou platí $S
\neq T$ a jejíž cena je optimální. Protože $S \neq T$, existují indexy $b_k, b_{k+1}$ takové, že
$s_{b_k} < s_{b_{k+1}}$. Ukážeme, že prohodíme-li tyto dva prvky posloupnosti (kterou si označíme
$T'$), její cena se nezvýší.

Označme si $u'(i)$ hodnotu $u(i)$ na takto vzniklé posloupnosti. Zřejmě platí, že $u'(i) = u(i)$ pro
$i \notin \{k, k+1\}$ -- pro $i < k$ se změní pouze pořadí indexů větší než $i$, ne součet jejich
$p$; pro $i > k+1$ pak $p_{b_k}, p_{b_{k+1}}$ nejsou ani v sumě obsaženy. Dále platí:

$$
\begin{array}{rclclcl}
u(k)    & = & s_{b_k}     & - & p_{b_{k+1}} & - & \sum_{j=k+2}^N p_{b_j} \\
u(k+1)  & = & s_{b_{k+1}} &   &             & - & \sum_{j=k+2}^N p_{b_j} \\
u'(k)   & = & s_{b_k}     &   &             & - & \sum_{j=k+2}^N p_{b_j} \\
u'(k+1) & = & s_{b_{k+1}} & - & p_{b_k}     & - & \sum_{j=k+2}^N p_{b_j} \\
\end{array}
$$

\noindent
Protože $s_{b_k} < s_{b_{k+1}}$, vyplývá z těchto rovnic, že $u'(k) < u(k+1)$ a také, že $u'(k+1) <
u(k+1)$. Takže $\max u'(i)$ nebude větší než $\max u(i)$, cena se nezvýší.

\medskip
\noindent
Prohozením těchto dvou prvků tedy získáme jeden ze tří případů:

\begin{itemize}
\item Cena takto změněné posloupnosti je menší. To je spor s optimalitou $T$.
\item Cena zůstala stejná a $S = T'$. To znamená, že i $S$ je optimální.
\item Cena zůstala stejná a $S \neq T'$. Pak opět můžeme prohodit dva prvky, což děláme tak dlouho,
      dokud nedostaneme jeden z předchozích dvou případů (protože provádíme de facto BubbleSort,
      stane se tak v konečném počtu kroků).
\end{itemize}

\noindent
Permutace závodníků $S$ je tedy optimálním řešením problému.
\hfill$\square$

\clearpage
\zadani{2}{3}

\noindent
Problém má optimální substrukturu, kterou můžeme popsat funkcí $v$, která udává optimální cenu
rozdělení $st$ studentů do tří skupin o kapacitách $s_1, s_2$ resp. $s_3$. První případ nám říká, že
jakékoliv rozdělení nula studentů má hodnotu nula. Druhý případ ošetřuje možnost, kdy se nějáká
skupina dostane na zápornou kapacitu. Význam třetího případu je ten, že optimální rozdělení vznikne
umístěním studenta $st$ do některé ze skupin a rekurzivní umístění ostatních studentů do ostatních
skupin s odpovídajícím způsobem sníženou kapacitou; ze tří možností vybíráme maximum. Cenou
optimálního řešení celého problému je potom $v\left(N,\frac{N}{3},\frac{N}{3},\frac{N}{3}\right)$,
výsledné rozdělení je možné získat analogickým postupem jakým počítáme cenu.

%uglyhack
\newlength{\xlen}\settowidth{\xlen}{$\max \{$}
$$v(st,s_1,s_2,s_3) = \begin{cases}
0       & \text{pokud } st = 0 \\
-\infty & \text{pokud } s_1 < 0 \vee s_2 < 0 \vee s_3 < 0 \\
\max \{ P[st,1] + v(st-1,s_1-1,s_2,s_3), & \\
        \hspace{\xlen} P[st,2] + v(st-1,s_1,s_2-1,s_3), & \text{jinak}\\
        \hspace{\xlen} P[st,3] + v(st-1,s_1,s_2,s_3-1)\} &
\end{cases}$$

\noindent
Můžeme si všimnout, že $st = s_1 + s_2 + s_3$, takže můžeme tento parametr v algoritmu vynechat.
Algoritmus na základě výše uvedené rovnosti odspoda vyplní trojrozměrnou tabulku $C$ s hodotami
$v(s_1,s_2,s_3)\ \forall 0 \leq s_1,s_2,s_3 \leq \frac{N}{3}$. V tabulce $G$ si pak uchovává
skupinu, do které je přiřazen student $s_1+s_2+s_3$ při těchto kapacitách skupin. To se děje na
řádcích 3 až 23. Tři cykly for zajistí postupné projití skupin podle vzrůstajícího $s_1+s_2+s_3$,
$C[s_1,s_2,s_3]$ se pak vypočte pomocí hodnot, které byly vypočítané předtím. Odpovídajícím způsobem
se vypočítá i hodnota v poli $G$. Protože je vnější cyklus omezen $N$ iteracemi a oba vnější
$\frac{N}{3}$ iteracemi, má tato část algoritmu časovou složitost $\O(N^3)$.

Na řádcích 24 až 35 se pak z tabulky $G$ vytvoří pole $Res$, kde $Res[st]$ obsahuje číslo skupiny,
do které patří student $st$. Tělo while cyklu se provede $N$-krát, protože součet $s_1+s_2+s_3$
konverguje k nule, složitost je tedy $\O(N)$.

\begin{algorithm}
\textsc{MakeSchedule}$(N,P)$
\begin{algorithmic}[1]
\STATE $C, G \la$ vytvoř trojrozměrné pole o velikosti $ N/3 \times N/3 \times N/3 $
\STATE $C[0,0,0] \la 0$
\FOR{$d \la 1$ to $N$}
\FOR{$s_1 \la 0$ to $\min(N/3,d)$}
\FOR{$s_2 \la 0$ to $\min(N/3,d-s_1)$}
\STATE $s_3 \la d-s_1-s_2$
\STATE $st \la s_1+s_2+s_3$
\IF{$s_3 \leq N/3$}
\STATE $C[s_1,s_2,s_3] \la -\infty$
\IF{$s_1 > 0 \wedge P[st,1] + C[s_1-1,s_2,s_3] > C[s_1,s_2,s_3]$}
\STATE $C[s_1,s_2,s_3] \la P[st,1] + C[s_1-1,s_2,s_3]$
\STATE $G[s_1,s_2,s_3] \la 1$
\ELSIF{$s_2 > 0 \wedge P[st,2] + C[s_1,s_2-1,s_3] > C[s_1,s_2,s_3]$}
\STATE $C[s_1,s_2,s_3] \la P[st,2] + C[s_1,s_2-1,s_3]$
\STATE $G[s_1,s_2,s_3] \la 2$
\ELSIF{$s_3 > 0 \wedge P[st,3] + C[s_1,s_2,s_3-1] > C[s_1,s_2,s_3]$}
\STATE $C[s_1,s_2,s_3] \la P[st,3] + C[s_1,s_2,s_3-1]$
\STATE $G[s_1,s_2,s_3] \la 3$
\ENDIF
\ENDIF
\ENDFOR
\ENDFOR
\ENDFOR

\STATE $s_1,s_2,s_3 \la N/3$
\STATE $Result \la$ vytvoř pole o délce $N$
\WHILE{$s_1+s_2+s_3 > 0$}
\STATE $Result[s_1+s_2+s_3] \la G[s_1,s_2,s_3]$
\IF{$G[s_1,s_2,s_3] = 1$}
\STATE $s_1 \la s_1 - 1$
\ELSIF{$G[s_1,s_2,s_3] = 2$}
\STATE $s_2 \la s_2 - 1$
\ELSE
\STATE $s_3 \la s_3 - 1$
\ENDIF
\ENDWHILE
\RETURN $Result$
\end{algorithmic}
\end{algorithm}

\clearpage
\zadani{2}{4}



\clearpage
\zadani{2}{5}

\noindent
Algoritmus ověří, zda je aktuálě zpracovávaný vrchol extremální a pokud není, znamená to, že kořen
alespoň jednoho podstromu má menší hodnotu. Na tento podstrom se algoritmus rekurzivně zavolá (pokud
mají menší hodnotu vrcholy obou podstromů, zavolá se na levý). Takto vždy dojde k extremálnímu
vrcholu -- viz níže.

%\begin{algorithm}
%\begin{algorithmic}
%\STATE $node \la root$
%\WHILE{not $\textsc{Leaf}(node)$}
%\IF{$\textsc{Value}(node.left) < \textsc{Value}(node)$}
%\STATE $node \la node.left$
%\ELSIF{$\textsc{Value}(node.right) < \textsc{Value}(node)$}
%\STATE $node \la node.right$
%\ELSE
%\RETURN $node$
%\ENDIF
%\ENDWHILE
%\RETURN $node$
%\end{algorithmic}
%\end{algorithm}

\begin{algorithm}
\textsc{FindExtremal}($node$)
\begin{algorithmic}
\IF{$\textsc{IsLeaf}(node)$}
\RETURN $node$
\ELSIF{$\textsc{Value}(node.left) < \textsc{Value}(node)$}
\RETURN \textsc{FindExtremal}($node.left$)
\ELSIF{$\textsc{Value}(node.right) < \textsc{Value}(node)$}
\RETURN \textsc{FindExtremal}($node.right$)
\ELSE
\RETURN $node$
\ENDIF
\end{algorithmic}
\end{algorithm}

Invariantem funkce \textsc{FindExtremal} je, že rodič aktuálně zpracovávaného vrcholu je větší než
on sám (nebo je zpracovávaný rodič kořen). V kořeni triviálně platí a rekurzivní volání na podstrom
s menší hodnotou kořene jej zachovávají. Pokud platí invariant a oba podstromy mají kořen s větší
hodnotou, je vrchol extremální. Podobně, pokud platí invariant a zpracovávaný vrchol je listem, je
tento list menší než všechny vrcholy a je tedy rovněž extremální.

Z toho, že se algoritmus rekurzivně volá na jeden z podstromů, kterýžto má o jedna menší výšku,
vyplývají následující fakta:

\begin{itemize}
\item Algoritmus je korektní, protože vždy dojde buď do listu nebo předtím narazí na vnitřní
extremální vrchol.
\item Algoritmus je konvergentní.
\item Protože se v každé instanci volá \textsc{Value} nejvýše čtyřikrát\footnote{počet těchto volání
by bylo možné snížit na dvě} a těchto instancí je nejvýše tolik, kolik je výška stromu, tedy $\log
n$, je počet volání $\O(\log n)$.
\end{itemize}

\hfill$\square$

\end{document}
