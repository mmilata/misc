\documentclass[12pt]{article}
\usepackage[czech]{babel}
\usepackage[utf8]{inputenc}
\usepackage{amsmath,amssymb}
\usepackage{enumerate}
\usepackage{hyperref}
\usepackage{algorithmic}
\usepackage{algorithm}
\usepackage{tikz}

%cisla stranek se muzou hodit
\pagestyle{plain}

\topmargin -5mm
\headsep 0mm
\headheight 3mm
\textheight 235mm
\textwidth 165mm
\oddsidemargin 0mm
\evensidemargin 0mm
\footskip 10mm

\newcommand{\eps}{\varepsilon}
\newcommand{\coL}{co\mbox{$-$}L}
\newcommand{\move}{\rightarrow}
\newcommand{\la}{\leftarrow}
\newcommand{\var}[1]{\textit{#1}}

\renewcommand{\O}{\mathcal{O}}

\newcommand{\zadani}[2]{
{\large
\noindent {\bf IB108 \hfill{} Sada #1, Příklad #2 \\[-4mm]}
\noindent\hrule
\vspace{2mm}
\noindent Vypracovali:\hfill{}Tomáš Krajča (255676), Martin Milata (256615)
\vspace{3mm}
\hrule
\bigskip\bigskip}
}


\begin{document}

\zadani{1}{1}

\noindent
Složitost třídícího algoritmu 1/3-Sort je v
$\mathcal{O}\left(n^{\log_{\frac{3}{2}}3}\right) \in
\mathcal{O}\left(n^{2.71}\right)$.

Nechť $n = j-i$ (velikost pole, které je vstupním parametrem funkce 1/3-Sort).
Lehce spočítáme, že velikost pole předávaná rekurzivně volaným 1/3-Sort je 
 $\le \frac{2}{3}n$. Z toho můžeme odvodit rekurentní rovnici:
\begin{eqnarray*}
T(n) & = & 3T\left(\left\lceil \frac{2}{3}n\right\rceil\right) + c\mbox{, kde }c \in
\mathbb{N}\\
T(1) & = & c'\mbox{, kde }c' \in \mathbb{N}
\end{eqnarray*}
Dle \textit{Master Theorem} je tedy $T(n) \in
\mathcal{O}\left(n^{\log_{\frac{3}{2}}3}\right)$, $(a = 3, b = \frac{3}{2}, d =
0)$.

Algoritmus korektně utřídí danou posloupnost.\\
Algoritmus konverguje - zřejmé, pro $i + 1 < j$ bude $k \ge 1$, tedy rekurzivně
volané funkce budou mít jako parametr alepoň o $1$ menší pole, než měla
původní funkce. Pro $i + 1 \ge j$ končí rekurzivní zanořování.\\
Algoritmus je parciálně korektní. \\
Invariant: V každém exempláři funkce $1/3-Sort$ je možné vstupní pole rozdělit
na následující $3$ části: $A[i]\dots A[i+k] \dots A[j-k] \dots A[j]$, kde $a)$
prostřední část je alespoň tak velká jako poslední část a zároveň $b)$ prostřední
část je neprázdná a zároveň $c)$ prostřední část je alespoň tak velká jako
první část.\\
$a)$
\begin{eqnarray*}
(j-k)-(i+k) =
j-k-k-i &\ge& k - 1  =
 j-(j-k+1)\\
 &\Longleftrightarrow&\\
 j-k-k-k-i+1 &\ge& 0\\
\left\lfloor\frac{3j-j+i-1-3i+3}{3}\right\rfloor -k -k &\ge& 0\\
 \left\lfloor\frac{2j-2i+2}{3}\right\rfloor
 -\left\lfloor\frac{j-i+1}{3}\right\rfloor -
 \left\lfloor\frac{j-i+1}{3}\right\rfloor &\ge& 0
\end{eqnarray*}
$b)$ předpokládáme, že $j\ge i+2$ (řádek $3$ algoritmu).
$$(j-k)-(i+k) = j-k-k-i \ge
(i+2)-\left\lfloor \frac{(i+2)-i+1}{3} \right\rfloor -\left\lfloor \frac{(i+2)-i+1}{3} \right\rfloor- i =
 0$$
 $c)$\\
 $$(j-k)-(i+k) \ge k-1 = (i+k-1)-i$$
 Pokračování na dalším listu.
 
 \newpage
\zadani{1}{1}
Nyní indukcí k délce vstupního pole ukážeme parciální korektnost algoritmu.\\
Pro pole délky $2$ (triviálně pro délku $1$): z řádků $1-4$ algoritmu je zřejmé,
že báze indukce platí (žádné rekurzivní zanořování).\\
Předpokládejme tedy, že daný algoritmus korektně utřídí pole délky $n-1$. Pak
volejme funkci $1/3-Sort(A,i,j):j-i=n$.
Dle indukčního předpokladu nám volání funkce $1/3-Sort$ na řádku $6$ utřídí část
pole $A[l]\dots A[j-k]$. Poté volání na řádku $7$ nám utřídí část pole
$A[i+k]\dots A[j]$. Přičemž bude
platit: $$A[i], A[i+1],\dots,A[j-k] \le A[j-k+1] \le A[j-k+2]
\dots \le A[j]$$ Plyne z invariantu z velikosti jednotlivých částí pole. Další
rekurzivní volání funkce $1/3-Sort$ na řádku $8$ nám setřídí zbývající část pole $A[i] \dots
A[j-k]$.
\hfill$\square$

\newpage
\zadani{1}{2}

\noindent
Nechť přiřazení trvá $1$ časovou jednotku, pak přesná časová složitost
algoritmu\\
je $\frac{1}{12}n+\frac{3}{8}n^2+\frac{5}{12}n^3+\frac{1}{8}n^4+1\in
\mathcal{O}(n^4)$.\\
Funkce \textit{UNKNOWN(n)} vypočítá hodnotu
$\frac{1}{12}n+\frac{3}{8}n^2+\frac{5}{12}n^3+\frac{1}{8}n^4$.
\\
Cyklus na řádku $4$ proběhne vždy $i$ krát ($(i+j)-j=i$). Protože $k \ge j$,
cyklus na řádku $5$ proběhne $i$ krát, $i-1$ krát $\dots 1$ krát (analyzujeme
pouze cykly na řádcích $4$, $5$). Tedy tato dvojice cyklů proběhne
$\frac{i}{2}\cdot (i+1)$. Cyklus na řádku $3$ proběhne $i$ krát, tedy dohromady
s~cykly na řádcích $4$, $5$ to bude $T'(i) = i\cdot \frac{i}{2}\cdot (i+1)$. Nyní už
snadno nahlédneme, že platí rekurentní vztahy (analyzujeme všechny
cykly):
\begin{eqnarray*}
T(1) &=& 1\\
T(n) &=& T(n-1)+T'(n)=T(n-1)+\frac{n}{2}\cdot\left(n+1\right)\cdot n\\
&\Longleftrightarrow&\\
T(n) &=&
(n+1)\left(\frac{1}{2}n+1\right)\left(2-5\left(\frac{1}{3}n+1\right)+3\left(\frac{1}{3}n+1\right)\left(\frac{1}{4}n+1\right)\right)\\
T(n) &=& \frac{1}{12}n+\frac{3}{8}n^2+\frac{5}{12}n^3+\frac{1}{8}n^4
\end{eqnarray*}

\newpage
\zadani{1}{3}

\noindent
Algoritmus:
\begin{algorithm}
\begin{algorithmic}[1]
\REQUIRE $A[l..r], A[~] \in \mathbb{Z}$
\ENSURE $\sum_{k=i}^{j}A[k]$ is maximum
\STATE	$max \leftarrow max\_suf \leftarrow 0$
\STATE  $i \leftarrow j \leftarrow i\_suf \leftarrow j\_suf \leftarrow 0$
\FOR{$k\leftarrow l$ to $r$}
\IF {$0 > max\_suf + A[k]$}
\STATE $max\_suf \leftarrow 0$
\STATE $i\_suf \leftarrow k+1$
\ELSE
\STATE $max\_suf \leftarrow max\_suf + A[k]$\
\STATE $j\_suf \leftarrow k$
\ENDIF
%\STATE $max\_suf \leftarrow MAX(0, max\_suf + A[k])$
\IF {$max < max\_suf$}
\STATE $max \leftarrow max\_suf$
\STATE $i \leftarrow i\_suf$
\STATE $j \leftarrow j\_suf$
\ENDIF
\ENDFOR
\IF {$max \le 0$}
\STATE $max = A[l]$
\STATE $i \leftarrow j \leftarrow l$
\FOR{$k\leftarrow l+1$ to $r$}
\IF {$A[k] > max$}
\STATE $max \leftarrow A[k]$
\STATE $i \leftarrow j \leftarrow k$
\ENDIF
\ENDFOR
\ENDIF
\RETURN $(i,j)$
\end{algorithmic}
\end{algorithm}

Analýza složitosti:\\
Nechť $n = r - l$ je velikost pole $A[~]$. Pak složitost algoritmu je zřejmě
$\Theta(n) \in \mathcal{O}\left(n\cdot \log n\right)$. Máme $2$ oddělené \textit{for}
cykly s $n$ iteracemi.

Pokračování na dalším listu.
\newpage
\zadani{1}{3}


\noindent
Konvergence triviální - pouze 2 konečné \textit{for} cykly.\\
Korektnost:\\ 
Dokážeme, že $max$ je suma podintervalu pole $A[~]$, která je maximální.
Argumentace, že nejlevější prvek takového podintervalu je $i$ a nejpravější $j$
by byla obdobná. Tento invariant platí po celou dobu výpočtu.\\

{
\renewcommand{\labelenumi}{$($\alph{enumi}$)$}
\begin{enumerate}

\item $\exists b.A[b] > 0$\\
Pro \textit{for} cyklus na řádku $3$ bude platit invariant: $max$ je 
suma podintervalu pole $A[l..k]$, která je maximální. Další invariant pro tento
\textit{for} cyklus je: $max\_suf$ je suma podintervalu pole $A[l..k]$, která je
maximální a zároveň nejpravější prvek podintervalu je $A[k]$. Pokud by měla být
proměnná $max$ nebo $max\_suf$ záporná, pokládáme je rovny $0$. To~můžeme,
protože víme, že v poli existuje prvek větší než $0$. Po opuštění tohoto
\textit{for} cyklu bude tedy v $max$ suma podintervalu pole $A[l..r]$, která je
maximální. Předpokládáme, že pole obsahuje nějaký prvek větší než $0$,
\textit{if} podmínka na řádku $17$ se tedy vyhodnotí na $false$.\\

\item $\forall b.A[b] \le 0$\\
Z úvodního předpokladu hned vidíme, že podinterval s maximální sumou bude
obsahovat $1$ prvek ($A[b] + A[c] \le A[b]$). Nechť tedy proběhne úvodních $16$
řádků, je zřejmé, že $max \le 0$, provedou se tedy řádky $18$..$26$. To je ovšem
aplikace standardního algoritmu pro nalezení největšího prvku pole $A[l..r]$.
Invariant pro \textit{for} cyklus na řádku $20$: $max$ je maximální položka
pole $A[l..k]$. Tento největší prvek pole bude jediný prvek maximálního
podintervalu pole $A[l..r]$.
\hfill$\square$

\end{enumerate}
}

\newpage
\zadani{1}{4}

Pro navržení datové struktury využijeme následující fakt. Mějme posloupnost
čísel (na~ob\-ráz\-ku 1--32) a nad ní postavený binární strom. Dále pak mějme nějaký
interval ležící v~této posloupnosti (na ob\-rá\-zku 19--27). Na obrázku jsou pro
názornost vyznačeny obě cesty z~kořene do~krajních bodů tohoto intervalu.
Vlastnost, která je pro nás podstatná je, že každá cesta z~kořene do nějákého
bodu intervalu vypadá následovně: projde uzlem, kde se cesty do~krajních
intervalů rozdvojují (v hraničních případech může být tento uzel kořen nebo
list) a pak pokračuje buď po cestě vedoucí do levého krajního bodu a z ní odbočí
doprava, nebo pokračují po cestě do pravého krajního bodu a z ní odbočí doleva,
případně dojdou po jedné z těchto cest až do krajního bodu.

\begin{center}
\begin{tikzpicture}
  [scale=0.9,
   level distance=10mm,
   emphedge/.style={draw=black,ultra thick},
   normedge/.style={draw=black,thin},
   every node/.style={inner sep=1pt,font=\scriptsize},
   selnode/.style={inner sep=1pt,minimum size=1.2mm,circle,draw=black,fill=black},
   level 1/.style={sibling distance=80mm,level distance=16mm},
   level 2/.style={sibling distance=40mm,level distance=14mm},
   level 3/.style={sibling distance=20mm,level distance=12mm},
   level 4/.style={sibling distance=10mm,level distance=10mm},
   level 5/.style={sibling distance=5mm,level distance=8mm,nodes={fill=none}}]
  \node[minimum size=0pt,inner sep=0pt] {}
  child {
    child {
      child {
        child {
          child { node {1} } child { node {2} }
        }
        child {
          child { node {3} } child { node {4} }
        }
      }
      child {
        child {
          child { node {5} } child { node {6} }
        }
        child {
          child { node {7} } child { node {8} }
        }
      }
    }
    child {
      child {
        child {
          child { node {9} } child { node {10} }
        }
        child {
          child { node {11} } child { node {12} }
        }
      }
      child {
        child {
          child { node {13} } child { node {14} }
        }
        child {
          child { node {15} } child { node {16} }
        }
      }
    }
  }
  %% middle
  child {
    edge from parent [emphedge]
    child {
      child {
        child {
          edge from parent [normedge]
          child { node {17} } child { node {18} }
        }
        child {
          child { node (node19) {19} } child { node (node20) {20} edge from parent [normedge] }
        }
      }
      child {
        node [selnode] {}
        edge from parent [normedge]
        child {
          child { node {21} } child { node {22} }
        }
        child {
          child { node {23} } child { node {24} }
        }
      }
    }
    child {
      child {
        child {
          node [selnode] {}
          edge from parent [normedge]
          child { node {25} } child { node {26} }
        }
        child {
          child { node (node27) {27} } child { node {28} edge from parent [normedge] }
        }
      }
      child {
        edge from parent [normedge]
        child {
          child { node {29} } child { node {30} }
        }
        child {
          child { node {31} } child { node {32} }
        }
      }
    }
  };
\node [selnode, above of=node19, node distance=2mm, shift={(0.55mm,0mm)}] {};
\node [selnode, above of=node20, node distance=2mm, shift={(-0.55mm,0mm)}] {};
\node [selnode, above of=node27, node distance=2mm, shift={(0.55mm,0mm)}] {};
\end{tikzpicture}
\end{center}

Jinak řečeno, projdou právě jednou uzlem, který je pravým (resp. levým) synem
nějákého uzlu ležícího na cestě do levého (resp. pravého) krajního
bodu, který sám na této cestě neleží (pokud není cílový bod zároveň bodem
krajním). Tyto uzly jsou na obrázku vyznačeny tečkou.

Jako datovou strukturu tedy použijeme binární strom, do jehož uzlů si budeme
ukládat přirozená čísla. Pro každý zadaný interval najdeme výše popsané uzly a
přičteme do nich číslo přiřazené danému intervalu. Tím dosáhneme toho, že na
každé cestě z kořene do listu bude ležet takový součet čísel, který odpovídá
součtu čísel přiřazených k intervalům, v nichž leží číslo v listu.

Pro nalezení uzlů, do kterých budeme ukládat hodnoty nám stačí projít obě cesty
z kořene do krajních bodů intervalu. Při tom projdeme nejvýše tolik uzlů, kolik
je dvojnásobek výšky stromu. Použijeme-li vyvážený binární strom, bude nám to
trvat logaritmický čas vzhledem k počtu listů a tedy vzhledem k velikosti
nosného intervalu.
% Použijeme-li vyvážený binární strom, stačí nám pro nalezení uzlů do kterých
% budeme ukládat hodnoty projít obě cesty do krajních bodů intervalu, což znamená
% projít počet uzlů rovnající se dvojnásobku výšky stromu, tedy vzhledem k počtu
% listů logaritmický.
Pro nalezení součtu náležícího nějákému číslu stačí
projít uzly od kořene k danému listu, což bude také trvat logaritmický čas.

Protože ve stromu uchováváme součty hodnot a protože parametry druhé operace
jsou pouze krajní body intervalu, musíme si ještě někde uchovávat jednotlivá
čísla přiřazená ke~každému intervalu. Pro tento účel použijeme dvourozměrné pole
velikosti $n \times n$ (kde $n$ má význam jako v zadání).

% \clearpage
% 
% Jako datovou strukturu použijeme vyvážený binární vyhledávací strom nad nosným
% intervalem. V každém uzlu bude uložené přirozené číslo $val$, které využijeme při
% implementaci zadaných operací.
% Ačkoli by to nebylo nutné, pro zjednodušení si do každého
% uzlu uložíme také přirozené číslo $i$, které nám v případě vnitřního uzlu bude
% říkat, jaká je hodnota nejpravějšího potomka levého podstromu, v případě listu
% pak přímo jeho hodnotu.
% 
% Pro uchování informace o tom jaké číslo je asociované s jakým intervalem dále
% použijeme dvourozměrné pole velikosti $n \times n$. První index bude značit
% počáteční bod intervalu, druhý koncový. Na začátku, kdy nebudou ve struktuře
% uložené žádné intervaly inicializujeme každý prvek pole nulou. Stav, kdy je k
% intervalu přiřazena nula a stav, kdy interval není ve struktuře obsažen jsou z
% hlediska obou operací nerozlišitelné.

\clearpage
\zadani{1}{4}

Algoritmus pro inicializaci datové struktury vytvoří vyvážený binární strom.
Hodnotu uzlu, nazvanou $val$, inicializuje na nulu. Pro zjednodušení si do uzlů přidáme
ještě položku $i$, která bude udávat hodnotu nejpravějšího potomka levého
podstromu (v případě vnitřního uzlu), v případě listu jeho hodnotu. Algoritmus
dále vytvoří pole velikosti $n \times n$, jehož prvky inicializuje na nulu.
Stav, kdy je k intervalu přiřazena nula a stav, kdy interval není ve~struktuře
obsažen jsou z hlediska obou operací nerozlišitelné.


\begin{algorithm}
\textsc{Init}$(n)$
\begin{algorithmic}[1]
\STATE $\var{Tree} \leftarrow \textsc{InitTree}(n,0)$
\STATE $\var{Arr} \leftarrow \mbox{alokuj pole velikosti }n\times n\mbox{ a vynuluj jeho prvky}$
\end{algorithmic}

\bigskip

\textsc{InitTree}$(\var{size},\var{offset})$
\begin{algorithmic}[1]
\STATE $\var{leftsize} \leftarrow \left\lfloor \var{size}/2 \right\rfloor$
\STATE $\var{newnode} \leftarrow \mbox{alokuj nový uzel}$
\STATE $\var{newnode.val} \leftarrow 0$
\STATE $\var{newnode.i} \leftarrow \var{leftsize} + \var{offset}$
\IF{$size = 1$}
\STATE $\var{newnode.left} \leftarrow \mbox{NIL}$
\STATE $\var{newnode.right} \leftarrow \mbox{NIL}$
\STATE $\var{newnode.i} \leftarrow newnode.i + 1$
\ELSE
\STATE $\var{newnode.left} \leftarrow \textsc{InitTree}(\var{leftsize,offset})$
\STATE $\var{newnode.right} \leftarrow
  \textsc{InitTree}(\var{size}-\var{leftsize,offset}+\var{leftsize})$
\ENDIF
\RETURN $\var{newnode}$
\end{algorithmic}
\end{algorithm}

Algoritmus pro nalezení součtu čísel přiřazených intervalům do kterých patří
dané číslo $k$ pouze sčítá hodnoty na cestě k listu s hodnotou $k$.
\begin{algorithm}
\textsc{Get}$(tree, k)$
\begin{algorithmic}[1]
\IF{$\var{tree} = \mbox{NIL}$}
\RETURN 0
\ELSIF{$k \leq \var{tree.i}$}
\RETURN $\textsc{Get}(\var{tree.left,k}) + \var{tree.val}$
\ELSE
\RETURN $\textsc{Get}(\var{tree.right,k}) + \var{tree.val}$
\ENDIF
\end{algorithmic}
\end{algorithm}

\clearpage
\zadani{1}{4}

Algoritmus pro změnu čísla nejprve přičte rozdíl nové a staré hodnoty do uzlů s
vlastností uvedenou na začátku a poté novou hodnotu zapíše do pole. Cesty k
hraničním uzlům se neprochází explicitně dvakrát, místo toho se zjišťuje vztah intervalu
který měníme a intervalů, které jsou reprezentovány oběma podstromy a podle toho
se buď algoritmus rekurzivně zavolá na daný podstrom, nebo je do kořene
podstromu přičten rozdíl hodnot, nebo se neprovede nic. Pro levý i pravý
podstrom se tedy zjišťuje, zda:

\begin{itemize}
\item Interval reprezentovaný daným podstromem je celý obsažen v měněném
intervalu. Pokud ano, je v jeho kořeni obsažena hodnota tohoto intervalu a
přičteme do něj tedy rozdíl nové a staré hodnoty (podmínky na řádcích 5 a 10).
\item Interval reprezentovaný daným podstromem je částečně obsažen v měněném
intervalu. Pokud ano, leží v něm hraniční bod intervalu, ke kterému se
rekurzivním voláním přiblížíme (podmínky na řádcích 7 a 12).
\end{itemize}

Konjunkce podmínek na řádcích 7 a 12, tedy situace, kdy levý krajní bod leží v
levém podstromě a pravý v pravém, a kdy dojde ke dvěma rekurzivním voláním, může
zřejmě nastat pouze jednou.

\begin{algorithm}
\textsc{Update}$(\var{int\_begin, int\_end, newvalue})$
\begin{algorithmic}[1]
\STATE
 \textsc{Update'}$(\var{Tree,0,n,int\_begin,int\_end,newvalue}-\var{Arr}[\var{int\_begin}][\var{int\_end}])$
\STATE $Arr[\var{int\_begin}][\var{int\_end}] \la \var{newvalue}$
\end{algorithmic}

\bigskip

\textsc{Update'}$(\var{node, begin, end, int\_begin, int\_end, updval})$
\begin{algorithmic}[1]
\STATE $\var{left\_begin} \leftarrow \var{begin}$
\STATE $\var{left\_end} \leftarrow \var{node.i}$
\STATE $\var{right\_begin} \leftarrow \var{left\_end+1}$
\STATE $\var{right\_end} \leftarrow \var{end}$
\IF{$\var{left\_begin} \geq \var{int\_begin} \wedge \var{left\_end} \leq \var{int\_end}$}
\STATE $\var{node.left.val} \la \var{node.left.val} + \var{updval}$
\ELSIF{$\var{int\_begin} \leq \var{left\_end}$}
\STATE $\textsc{Update'}(\var{node.left, left\_begin, left\_end, int\_begin, int\_end, updval})$
\ENDIF

\IF{$\var{right\_begin} \geq \var{int\_begin} \wedge \var{right\_end} \leq \var{int\_end}$}
\STATE $\var{node.right.val} \la \var{node.right.val} + \var{updval}$
\ELSIF{$\var{right\_begin} \leq \var{int\_end}$}
\STATE $\textsc{Update'}(\var{node.right, right\_begin, right\_end, int\_begin, int\_end, updval})$
\ENDIF
\end{algorithmic}
\end{algorithm}
\hfill$\square$

\clearpage
\zadani{1}{5}

\noindent
Rozdělme kredity následovně (cena je vyjádřena jako počet přístupů do pole):\\

\noindent
\begin{tabular}{|c|c|c|}
\hline
Operace & Kredity & Cena \\
\hline
\textsc{Přidej}  & 1  & 1      \\
\textsc{Sečti}   & 4  & 3      \\
\textsc{Odstraň-Maximum} & 0  & cena shora ohraničena hodnotou odstraňovaného
prvku \\
\hline
\end{tabular}

\vspace{4mm}
\noindent
Při každém použití operace \textsc{Sečti} přidáme na účet jeden kredit.
Získáváme tak invariant, že každé číslo $n$ v multimnožině má na účtu $n-1$
kreditů. Jinými slovy, každému indexu pole $i$ (s hodnotou $Arr[i]$) je
přiřazeno $Arr[i] \cdot (i-1)$ kreditů. Důkaz indukcí podle indexu $i$:
\begin{description}
\item[Báze] ($i=1$) Nula kreditů -- triviálně platí.
\item[Indukční krok] Předpokládejme, že pro $i=1 \ldots n$ tvrzení platí.
Pak každé číslo $n+1$ v multimnožině muselo vzniknout součtem dvou čísel $1 \le
a,b \le n$, takže bude mít přiřazen součet jejich kreditů a navíc jeden kredit za
operaci \textsc{Sečti}. Tedy $(a - 1) + (b - 1) + 1 = a + b - 1 = (n + 1) - 1 =
n$ kreditů.
\end{description}

Cena operace \textsc{Odstraň-Maximum} je v nejhorším případě rovna hodnotě
odstraňovaného maxima. Takový případ nastane, pokud je aktuální maximum $n$
jediným prvkem multimnožiny a po jeho odstranění se musí prohledat všechny
indexy menší než $n$, aby se určilo nové maximum. Protože má $n$ na účtu $n-1$
kreditů, můžeme zaplatit prohledání těchto $n-1$ indexů, aniž bychom uvedli účet
do záporného stavu.

V případě, že jsou v multimnožině ještě nějaké prvky, může být skutečná cena
operace menší. To, že nám nějaké kredity zbudou ovšem není podstatné.

Posloupnost $n$ operací bude stát nejvýše $4n$ kreditů a je tím pádem v $\O(n)$.
\hfill$\square$

\newpage
\zadani{1}{6}

\begin{enumerate}
\item Tvrzení neplatí.\\
Protipříklad: Nechť $n \in \mathbb{N}$ je vhodný, velký počet operací. Uděláme
$\frac{n}{2}$ operací \textsc{INSERT(S,i)}, poté budeme už 
jen střídat operace \textsc{MIN-ALL(S)} a \textsc{INSERT(S,i+1)} (v~tomto
pořadí). Takže uděláme
$\frac{n}{2} + \frac{n}{4}$ operací \textsc{INSERT(S,\_)} a $\frac{n}{4}$
operací \textsc{MIN-ALL(S)}, které odeberou jen prvek \textsc{i+1} (cena bude
$\frac{n}{2}+1$). Tedy časová složitost těchto $n$ operací bude $\frac{3n}{4}\cdot 1 + \frac{n}{4}\cdot
\left(\frac{n}{2}+1\right) \in \Theta\left(n^2\right)$.

\item Tvrzení platí.\\
Analyzujeme technikou účtů: Každá operace \textsc{INSERT(S,\_)} si přinese $3$
kredity. $1$~kredit se použije na zaplacení operace vložení, $1$~kreditem se
zaplatí (pokud to bude třeba) přístup na prvek při operaci \textsc{MIN-ONE(S)}. \textsc{MIN-ONE(S)} si
nepřinese žádný kredit. Invariantem tedy bude:
na účtu máme alespoň $2$ krát tolik kreditů, kolik máme prvků v seznamu $S$. Protože
operace \textsc{MIN-ONE(S)} odstraní ze seznamu alespoň $1$~prvek (jinak by
 byla nezajímavá - stála konstantní cenu), můžeme jeho $1$ zbylý
kredit vložit na~účet jednoho neodstraněného prvku (ten na svém účtu už alespoň
$1$ kredit mít musel). Složitost $n$ operací tedy bude
$\mathcal{O}(n)$.\hfill$\square$
\end{enumerate}


\newpage
\zadani{1}{9}

\begin{description}
\item[Horní odhad] Horní odhad provedeme konstrukcí a analýzou složitosti (ne
nezbytně optimálního) algoritmu. Nechť $n$ značí počet disků se kterými algoritmus
pracuje (na~po\-čá\-tku roven celkovému počtu disků), $m$ pozici aktuálně největšího
disku.\\

%\begin{algorithm}
\begin{algorithmic}
\IF {$n = 1$}
\RETURN
\ELSE
\STATE obrať $m$ disků (takže největší disk bude nahoře)
\STATE obrať $n$ disků (největší disk bude dole)
\STATE rekurzivně zavolej tento algoritmus na $n-1$ horních disků
\ENDIF
\end{algorithmic}
%\end{algorithm}

Algoritmus využívá toho, že (podobně jako u zásobníku) můžeme pracovat jen s
několika horními disky bez ohledu na to, co je pod nimi. Algoritmus je zřejmě
konvergentní (rekurze se snižujícím se $n$) a korektně uspořádá disky
(indukcí -- triviálně pro jeden disk, pro $n+1$ disků nejprve přesune největší
disk dospoda a potom uspořádá zbylých $n$ disků). Přitom provede $2n - 2$
operací ,,obrať``.
% což můžeme ukázat indukcí.
% \begin{description}
% \item[Báze] Pro jeden prvek je uspořádání triviálně splněno.
% \item[Indukční krok] Předpokládejme, že algoritmus korektně uspořádá $n$ disků.
% Pak přemístěním největšího disku dolů a následným uspořádáním $n$ zbývajících
% disků dostaneme $n+1$ korektně uspořádaných disků.
% \end{description}

\item[Dolní odhad] Pro provedení dolního odhadu si stačí uvědomit tyto
skutečnosti:
\begin{itemize}
\item Cílové uspořádání disků má tu vlastnost, že každá dvojice vzájemně
sousedících disků se velikostí liší o jedničku. Tedy pro $n$ disků existuje
$n-1$ sousedících dvojic takových, že se liší právě o jedna.
\item Provedením jedné operace se počet dvojic takových, které se liší o
jedničku změní nejvýše o jednu takovou dvojici (Obrátíme-li $m$ disků, zrušíme
% vysvetlit lepe? tady m znaci pozici, dole zase velikost disku ...
dvojici $m, m+1$ a vznikne nám dvojice $1, m+1$, ostatním diskům zůstanou
původní sousedé).
\end{itemize}

Z těchto pozorování vyplývá, že máme-li $n$ disků v pořadí
$$1,n,2,n-1,3,n-2,\ldots,\frac{n}{2},\frac{n}{2}+1 \mbox{ (pro sudá $n$)}$$
nebo
$$1,n,2,n-1,3,n-2,\ldots,\left\lceil\frac{n}{2}\right\rceil+1,\left\lceil\frac{n}{2}\right\rceil
\mbox{ (pro lichá $n$),}$$
tedy takovém, kdy pouze jedna sousedící dvojice je ve správném pořadí,
potřebujeme provést alespoň $n-2$ operací otočení, abychom mohli dosáhnout toho,
že budou disky seřazené podle velikosti.

\end{description}

\noindent
Dostáváme tedy $\mathcal{A} \in \O(n)$ a zároveň $\mathcal{A} \in \Omega(n)$, což
mimo jiné znamená, že algoritmus z~horního odhadu je asymptoticky optimální.
\hfill$\square$

\end{document}

